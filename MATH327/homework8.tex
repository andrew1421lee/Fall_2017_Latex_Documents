\documentclass{article}
\usepackage{amsthm, amsmath, pgfplots, pgfplotstable, enumitem}
\usepackage[margin=0.5in]{geometry}
\setlength\parindent{0pt}
\begin{document}
\noindent\textbf{Math 327 Homework 8}\hfill Anchu A. Lee\\
\noindent\today\\\newline
\textbf{Question 9.2:}
An electrical firm manufactures light bulbs that
have a length of life that is approximately normally
distributed with a standard deviation of 40 hours. If
a sample of 30 bulbs has an average life of 780 hours,
find a 96\% confidence interval for the population mean
of all bulbs produced by this firm.
\begin{description}
    \item $96 = 100(1-\alpha)$, so $\alpha = 0.04$ and 
    $z_{0.02}=2.05$. $\sigma = 40$, $\bar{x} = 780$, $n = 30$. \\
    $780 - (2.05)(\frac{40}{\sqrt{30}}) < \mu < 780 + (2.05)
    (\frac{40}{\sqrt{30}})$ \\ \boldmath$765.02 < \mu < 794.97$
\end{description}
\textbf{Question 9.4:}
The heights of a random sample of 50 college students 
showed a mean of 174.5 centimeters and a standard 
deviation of 6.9 centimeters.
\begin{enumerate}[label = (\alph*) ]
    \item Construct a 98\% confidence interval for the mean
    height of all college students.
    \begin{description}
        \item $n = 50$, $\bar{x}=174.5$, $\sigma = 6.9$, 
        $98 = 100(1-\alpha)$, $\alpha = 0.02$ and $z_{0.01}=2.33$\\
        $174.5-(2.33)(\frac{6.9}{\sqrt{50}}) < \mu < 174.5+(2.33)
        (\frac{6.9}{sqrt{50}})$\\ \boldmath$172.23 < \mu < 176.77$
    \end{description}
    \item  What can we assert with 98\% confidence about the
    possible size of our error if we estimate the mean
    height of all college students to be 174.5 centimeters?
    \begin{description}
        \item \boldmath$error < (2.33)(\frac{6.9}{\sqrt{50}}) = 2.27$
    \end{description}
\end{enumerate}
\textbf{Question 9.6:}
How large a sample is needed in Exercise 9.2 if we
wish to be 96\% confident that our sample mean will be
within 10 hours of the true mean?
\begin{description}
    \item $10 = (2.05)(\frac{40}{\sqrt{n}})$, \boldmath$n = 67.24 \approx 68$.
\end{description}
\textbf{Question 9.10:}
A random sample of 12 graduates of a certain
secretarial school typed an average of 79.3 words per
minute with a standard deviation of 7.8 words per
minute. Assuming a normal distribution for the number 
of words typed per minute, find a 95\% confidence
interval for the average number of words typed by all
graduates of this school.
\begin{description}
    \item $n = 12$, $\bar{x} = 79.3$, $\sigma = 7.8$, $95 = 100(1-\alpha)$, 
    $\alpha=0.05$, $t_{0.025}=2.201$\\
    $79.3 - (2.201)(\frac{7.8}{\sqrt{12}}) < \mu < 79.3 + (2.201)(\frac{7.8}
    {\sqrt{12}})$\\\boldmath$74.34 < \mu < 84.26$
\end{description}
\textbf{Question 9.12:}
A random sample of 10 chocolate energy bars of
a certain brand has, on average, 230 calories per bar,
with a standard deviation of 15 calories. Construct a
99\% confidence interval for the true mean calorie 
content of this brand of energy bar. Assume that the 
distribution of the calorie content is approximately 
normal.
\begin{description}
    \item $n=10$, $\bar{x}=230$, $\sigma=15$, $99=100(1-\alpha)$, $\alpha=0.01$,
    $t_{0.005}=3.250$\\ $230-(3.250)(\frac{15}{\sqrt{10}}) < \mu < 230+(3.250)
    (\frac{15}{\sqrt{10}})$\\\boldmath$214.58 < \mu < 245.42$
\end{description}
\textbf{Question 9.36:}
Two kinds of thread are being compared for
strength. Fifty pieces of each type of thread are tested
under similar conditions. Brand A has an average 
tensile strength of 78.3 kilograms with a standard 
deviation of 5.6 kilograms, while brand B has an average
tensile strength of 87.2 kilograms with a standard 
deviation of 6.3 kilograms. Construct a 95\% confidence
interval for the difference of the population means.
\begin{description}
    \item $n_A=50$, $\bar{x}_A=78.3$, $\sigma_A=5.6$, $n_B=50$, $\bar{x}_B=87.2$, 
    $\sigma_B=6.3$, $95 = 100(1-\alpha)$, $\alpha = 0.05$, $z_{0.025}=1.96$\\
    $(87.2-78.3)-(1.96)(\sqrt{\frac{5.6^2}{50}+\frac{6.3^2}{50}}) < \mu_A - \mu_B
    < (87.2-78.3)+(1.96)(\sqrt{\frac{5.6^2}{50}+\frac{6.3^2}{50}})$\\
    \boldmath$6.56 < \mu_A - \mu_B < 11.24$
\end{description}
\textbf{Question 9.38:}
Two catalysts in a batch chemical process, are
being compared for their effect on the output of the
process reaction. A sample of 12 batches was prepared
using catalyst 1, and a sample of 10 batches was 
prepared using catalyst 2. The 12 batches for which 
catalyst 1 was used in the reaction gave an average yield
of 85 with a sample standard deviation of 4, and the
10 batches for which catalyst 2 was used gave an average 
yield of 81 and a sample standard deviation of 5.
Find a 90\% confidence interval for the difference 
between the population means, assuming that the 
populations are approximately normally distributed with
equal variances.
\begin{description}
    \item $n_1 = 12$, $\bar{x}_1=85$, $\sigma_1=4$, $n_2=10$, $\bar{x}_2=81$,
    $\sigma_2=5$, $s_p=\sqrt{\frac{(12-1)4^2+(10-1)5^2}{12+10-2}}=4.47$, $90=100(1-\alpha)$, 
    $\alpha=0.1$, $t_{1,0.05}=1.725$\\
    $(85-81)-(1.725)(4.47)(\sqrt{\frac{1}{12}+\frac{1}{10}}) < \mu_1 - \mu_2 <
    (85-81)+(1.725)(4.47)(\sqrt{\frac{1}{12}+\frac{1}{10}}) $\\
    \boldmath$0.7 < \mu_1 - \mu_2 < 7.3$
\end{description}
\textbf{Question 9.40:}
In a study conducted at Virginia Tech on the
development of ectomycorrhizal, a symbiotic relationship 
between the roots of trees and a fungus, in which
minerals are transferred from the fungus to the trees
and sugars from the trees to the fungus, 20 northern
red oak seedlings exposed to the fungus $Pisolithus$ $tinctorus$ 
were grown in a greenhouse. All seedlings were planted in 
the same type of soil and received the same amount of 
sunshine and water. Half received no nitrogen at planting 
time, to serve as a control, and the other half received 
368 ppm of nitrogen in the form NaNO$_3$ . The stem weights, 
in grams, at the end of 140 days were recorded as follows:
\begin{center}
\begin{tabular}{ c  c }
    No Nitrogen & Nitrogen\\
    \hline
    0.32 & 0.26 \\ 
    0.53 & 0.43 \\
    0.28 & 0.47 \\
    0.37 & 0.49 \\
    0.47 & 0.52 \\
    0.43 & 0.75 \\
    0.36 & 0.79 \\
    0.42 & 0.86 \\
    0.38 & 0.62 \\
    0.43 & 0.46 \\
    \hline
\end{tabular}
\end{center}
Construct a 95\% confidence interval for the difference
in the mean stem weight between seedlings that receive 
no nitrogen and those that receive 368 ppm of nitrogen. 
Assume the populations to be normally distributed with 
equal variances.
\begin{description}
    \item $n_1=10$, $\bar{x}_1=0.399$, $\sigma_1=0.073$, $n_2=10$, $\bar{x}_2=0.565$, 
    $\sigma_2=0.187$, $s_p=\sqrt{\frac{(10-1)0.073^2+(10-1)0.187^2}{10+10-2}}=0.142$, 
    $95=100(1-\alpha)$, $\alpha=0.05$, $t_{0.025}=2.101$\\
    $(0.565-0.399)-(2.101)(0.142)(\sqrt{\frac{1}{10}+\frac{1}{10}}) < \mu_1-\mu_2 <
    (0.565-0.399)+(2.101)(0.142)(\sqrt{\frac{1}{10}+\frac{1}{10}})$\\
    \boldmath$0.033 < \mu_1 - \mu_2 < 0.299$
\end{description}
\textbf{Qustion 9.42:}
An experiment reported in Popular Science
compared fuel economies for two types of similarly
equipped diesel mini-trucks. Let us suppose that 12
Volkswagen and 10 Toyota trucks were tested in 90 
kilometer-per-hour steady-paced trials. If the 12 
Volkswagen trucks averaged 16 kilometers per liter with a
standard deviation of 1.0 kilometer per liter and the 10
Toyota trucks averaged 11 kilometers per liter with a
standard deviation of 0.8 kilometer per liter, construct
a 90\% confidence interval for the difference between the
average kilometers per liter for these two mini-trucks.
Assume that the distances per liter for the truck models 
are approximately normally distributed with equal
variances.
\begin{description}
    \item $n_V=12$, $\bar{x}_V=16$, $\sigma_V=1$, $n_T=10$, $\bar{x}_T=11$, $\sigma_T=0.8$,
    $s_p=\sqrt{\frac{(12-1)1^2+(10-1)0.8^2}{12+10-2}}=0.92$, $90=100(1-\alpha)$, $\alpha=0.1$,
    $t_{0.05}=1.725$\\
    $(16-11)-(1.725)(0.92)(\sqrt{\frac{1}{12}+\frac{1}{10}}) < \mu_V-\mu_T < 
    (16-11)+(1.725)(0.92)(\sqrt{\frac{1}{12}+\frac{1}{10}})$\\
    \boldmath$4.3 < \mu_V-\mu_T < 5.68$
\end{description}
\textbf{Question 9.44:}
Referring to Exercise 9.43, find a 99\% confidence
interval for $\mu_1 - \mu_2$ if tires of the two brands are 
assigned at random to the left and right rear wheels of
8 taxis and the following distances, in kilometers, are
recorded
\begin{center}
    \begin{tabular}{c c c}
        Taxi  & Brand A & Brand B\\
        \hline
        1 & 34,400 & 36,700 \\
        2 & 45,500 & 46,800 \\
        3 & 36,700 & 37,700 \\
        4 & 32,000 & 31,100 \\
        5 & 48,400 & 47,800 \\
        6 & 32,800 & 36,400 \\
        7 & 38,100 & 38,900 \\
        8 & 30,100 & 31,500 \\
    \end{tabular}
\end{center}
Assume that the differences of the distances are 
approximately normally distributed.\\\newline
\textbf{Question 9.46:}
The following data represent the running times
of films produced by two motion-picture companies.
\begin{center}
\begin{tabular}{c | l}
    Company & Time (minutes)\\
    \hline
    I & 103 94 110 87 98 \\
    II & 97 82 123 92 175 88 118
\end{tabular}
\end{center}
Compute a 90\% confidence interval for the difference
between the average running times of films produced by
the two companies. Assume that the running-time 
differences are approximately normally distributed with
unequal variances.\\\newline
\textbf{Question 9.48:}
An automotive company is considering two
types of batteries for its automobile. Sample information 
on battery life is collected for 20 batteries of
type A and 20 batteries of type B. The summary
statistics are $\bar{x}_A = 32.91$, $\bar{x}_B = 30.47$, $s_A = 1.57$,
and $s_B = 1.74$. Assume the data on each battery are
normally distributed and assume $\sigma_A = \sigma_B$\\\newline
\textbf{Question 9.50:}
Two levels (low and high) of insulin doses are
given to two groups of diabetic rats to check the 
insulinbinding capacity, yielding the following data:
\begin{center}
\begin{tabular}{c c c c}
    Low dose: & $n_1 = 8$ & $\bar{x}_1 = 1.98$ & $s_1 = 0.51$ \\
    High dose: & $n_2 = 13$ & $\bar{x}_2 = 1.30$ & $s_2 = 0.35$
\end{tabular}
\end{center}
Assume that the variances are equal. Give a 95\% 
confidence interval for the difference in the true average
insulin-binding capacity between the two samples.\\\newline
\textbf{Question 9.92:}
A study was undertaken at Virginia Tech to determine if 
fire can be used as a viable management tool to increase 
the amount of forage available to deer during the critical 
months in late winter and early spring. Calcium is a 
required element for plants and animals. The amount taken 
up and stored in plants is closely correlated to the amount 
present in the soil. It was hypothesized that a fire may 
change the calcium levels present in the soil and thus 
affect the amount available to deer. A large tract of land 
in the Fishburn Forest was selected for a prescribed burn. 
Soil samples were taken from 12 plots of equal area just 
prior to the burn and analyzed for calcium. Postburn 
calcium levels were analyzed from the same plots. These 
values, in kilograms per plot, are presented in the 
following table:
\begin{center}
\begin{tabular}{c c c}
    & \multicolumn{2}{c}{Calcium Level (kg/plot)}\\
    \hline
    Plot & Preburn & Postburn \\
    \hline
    1  & 50 & 9  \\
    2  & 50 & 18 \\
    3  & 82 & 45 \\
    4  & 64 & 18 \\
    5  & 82 & 18 \\
    6  & 73 & 9  \\
    7  & 77 & 32 \\
    8  & 54 & 9  \\
    9  & 23 & 18 \\
    10 & 45 & 9  \\
    11 & 36 & 9  \\
    12 & 54 & 9  \\
\end{tabular}
\end{center}
Construct a 95\% confidence interval for the mean 
difference in calcium levels in the soil prior to and 
after the prescribed burn. Assume the distribution of 
differences in calcium levels to be approximately normal.
\\\newline
\end{document}