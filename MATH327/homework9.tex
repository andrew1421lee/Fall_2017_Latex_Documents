\documentclass{article}
\usepackage{amsthm, amsmath, pgfplots, pgfplotstable, enumitem, textcomp}
\usepackage[margin=0.5in]{geometry}
\setlength\parindent{0pt}
\begin{document}
\noindent\textbf{Math 327 Homework 9}\hfill Anchu A. Lee\\
\noindent\today\\\newline
\textbf{Question 9.54:}
A manufacturer of MP3 players conducts a set
of comprehensive tests on the electrical functions of its
product. All MP3 players must pass all tests prior to
being sold. Of a random sample of 500 MP3 players, 15
failed one or more tests. Find a 90\% confidence interval
for the proportion of MP3 players from the population
that pass all tests.
\begin{description}
    \item \textbf{Answer:} $n=500$, $\hat{p}=485 / 500 = 0.97$, $\hat{q}=0.03$, $z_{0.05}=1.645$\newline
    $1.645\cdot \sqrt{\frac{0.97\cdot 0.03}{500}} = 0.013$, $0.97 - 0.013 < p < 0.97 + 0.013$\newline
    \boldmath{$0.957 < p < 0.983$}
\end{description}

\textbf{Question 9.56:}
A geneticist is interested in the proportion of
African males who have a certain minor blood disorder. 
In a random sample of 100 African males, 24 are
found to be afflicted.
\begin{enumerate}[label = (\alph*) ]
    \item Compute a 99\% confidence interval for the 
    proportion of African males who have this blood disorder.
    \item What can we assert with 99\% confidence about the
    possible size of our error if we estimate the 
    proportion of African males with this blood disorder to be
    0.24?
\end{enumerate}
\begin{description}
    \item \textbf{Answer:} $n=100$, $\hat{p}=24/100 = 0.24$, $\hat{q}=0.76$, $z_0.005=2.575$\newline
    $2.575\cdot \sqrt{\frac{0.24\cdot 0.76}{100}}=0.110$ \newline
    (a) \boldmath$0.130 < p < 0.350$\newline
    (b) \boldmath$0.110$
\end{description}

\textbf{Question 9.64:}
A study is to be made to estimate the proportion of 
residents of a certain city and its suburbs who
favor the construction of a nuclear power plant near
the city. How large a sample is needed if one wishes to
be at least 95\% confident that the estimate is within
0.04 of the true proportion of residents who favor the
construction of the nuclear power plant?
\begin{description}
    \item \textbf{Answer:} $e=0.04$, $z_{0.025}=1.96$\newline
    $n=\frac{(1.96)^2}{(4\cdot 0.04)^2} \approx $ \boldmath$601$
\end{description}

\textbf{Question 9.66:}
Ten engineering schools in the United States
were surveyed. The sample contained 250 electrical
engineers, 80 being women; 175 chemical engineers, 40
being women. Compute a 90\% confidence interval for
the difference between the proportions of women in
these two fields of engineering. Is there a significant
difference between the two proportions?
\begin{description}
    \item \textbf{Answer:} $n_1=250$, $\hat{p}_1=80/250=0.32$, $n_2=175$, $\hat{p}_2=40/175=0.2286$, $z_{0.05}=1.645$\newline
    $(0.32-0.2286)\pm(1.645)\sqrt{\frac{0.32\cdot 0.68}{250}+\frac{0.2286\cdot 0.7714}{175}}$\newline
    $0.0914\pm 0.713$, \newline
    \boldmath$0.0201 < p_1 - p_2 < 0.1627$. Yes, there are significantly more women
    in electrical engineering than in chemical engineering.
\end{description}

\textbf{Question 9.68:}
In the study \textit{Germination and Emergence of
Broccoli}, conducted by the Department of Horticulture
at Virginia Tech, a researcher found that at 5\textdegree C, 10
broccoli seeds out of 20 germinated, while at 15\textdegree C, 15
out of 20 germinated. Compute a 95\% confidence interval 
for the difference between the proportions of germination 
at the two different temperatures and decide if there is a 
significant difference.
\begin{description}
    \item \textbf{Answer:} $n_1=20$, $\hat{p}_1=0.50$, $n_2=20$, $\hat{p}_2=0.75$, $z_{0.025}=1.96$\newline
    $(0.5-0.75)\pm (1.96)\sqrt{\frac{0.5\cdot 0.5}{20} + \frac{0.75\cdot 0.25}{20}}$\newline
    $-0.25\pm 0.2899$\newline
    \boldmath$-0.5399 < p_1 - p_2 < 0.0399$, The interval includes 0 so 
    significance cannot be shown.
\end{description}

\textbf{Question 9.70:}
According to \textit{USA Today} (March 17, 1997),
women made up 33.7\% of the editorial staff at local
TV stations in the United States in 1990 and 36.2\% in
1994. Assume 20 new employees were hired as editorial
staff.
\begin{enumerate}[label = (\alph*) ]
    \item Estimate the number that would have been women
    in 1990 and 1994, respectively.
    \item  Compute a 95\% confidence interval to see if there
    is evidence that the proportion of women hired as
    editorial staff was higher in 1994 than in 1990.
\end{enumerate}
\begin{description}
    \item \textbf{Answer:} $n_1=20$, $\hat{p}_1=0.337$, $n_2=20$, $\hat{p}_2=0.361$\newline
    (a) $n_1\hat{p}_1=20\cdot 0.337 \approx 7$, $n_2\hat{p}_2=20\cdot 0.362\approx 7$.\newline
    (b) $z_{0.025}=1.96$\newline
    $(0.337-0.362)\pm(1.96)\sqrt{\frac{0.337\cdot 0.663}{20}+\frac{0.362\cdot 0.638}{20}}$\newline
    $-0.025\pm 0.295$\newline
    \boldmath$-0.320<p_1-p_2<0.270$. No evidence that the proportion of women hired was higher in 1994.
    Includes 0.
\end{description}

\textbf{Question 9.72:}
A random sample of 20 students yielded a mean
of $\bar{x} = 72$ and a variance of $s^2 = 16$ for scores on a
college placement test in mathematics. Assuming the
scores to be normally distributed, construct a 98\% 
confidence interval for $\sigma^2$.
\begin{description}
    \item \textbf{Answer:} $s^2=16$, $v=19$, $\chi_{0.01}^2=36.191$, $\chi_{0.99}^2=7.633$\newline
    $\frac{19\cdot 16}{36.191} < \sigma^2 < \frac{19\cdot 16}{7.633}$\newline
    \boldmath$8.4 < \sigma^2 < 39.827$
\end{description}

\textbf{Question 9.78:}
Construct a 90\% confidence interval for $\sigma^2_1 / \sigma^2_2$ 
in Exercise 9.43 on page 295. Were we justified in assuming that 
$\sigma^2_1 \not= \sigma^2_2$ when we constructed the confidence
interval for $\mu_1 - \mu_2$?
\begin{description}
    \item \textbf{Answer:} $s_1^2=5000^2$, $s_2^2=6100^2$, $f_{0.05}(11,11)=2.82$.\newline
    $(\frac{5000}{6100})^2\cdot \frac{1}{2.82} < \frac{\sigma_1^2}{\sigma_2^2} < (\frac{5000}{6100})^2\cdot(2.82)$\newline
    \boldmath$0.238<\frac{\sigma_1^2}{\sigma_2^2}<1.895$. The interval contains 1 so it was not reasonable to assume $\sigma_1^2\not=\sigma_2^2$
\end{description}

\textbf{Question 9.90:}
According to the \textit{Roanoke Times}, McDonald’s
sold 42.1\% of the market share of hamburgers. A random 
sample of 75 burgers sold resulted in 28 of them
being from McDonald’s. Use material in Section 9.10
to determine if this information supports the claim in
the \textit{Roanoke Times}.
\begin{description}
    \item \textbf{Answer:} $n=75$, $x=28$, $\hat{p}=28/75=0.3733$, $z_{0.025}=1.96$\newline
    $0.3733\pm (1.96)\sqrt{\frac{0.3733\cdot 0.6267}{75}} = 0.3733\pm 0.1095$\newline
    \boldmath$0.2638<p<0.4828$. Interval contains 0.421 so claim is reasonable.
\end{description}

\textbf{Question 9.96:}
An anthropologist is interested in the proportion
of individuals in two Indian tribes with double 
occipital hair whorls. Suppose that independent samples are
taken from each of the two tribes, and it is found that
24 of 100 Indians from tribe A and 36 of 120 Indians
from tribe B possess this characteristic. Construct a
95\% confidence interval for the difference $p_B − p_A$ 
between the proportions of these two tribes with occipital
hair whorls.
\begin{description}
    \item \textbf{Answer:} $n_A=100$, $\hat{p}_A=24/100=0.24$, $n_B=120$, $\hat{p}_B=36/120=0.30$, $z_{0.025}=1.96$\newline
    $(0.30-0.24)\pm(1.96)\sqrt{\frac{0.24\cdot 0.76}{100}+\frac{0.3\cdot 0.7}{120}}=0.06\pm 0.117$\newline
    \boldmath$-0.057< p_B-p_A < 0.177$.
\end{description}

\textbf{Question 9.106:}
A random sample of 30 firms dealing in wireless
products was selected to determine the proportion of
such firms that have implemented new software to 
improve productivity. It turned out that 8 of the 30 had
implemented such software. Find a 95\% confidence 
interval on $p$, the true proportion of such firms that have
implemented new software.
\begin{description}
    \item \textbf{Answer:} $n=30$, $x=8$, $z_{0.025}=1.96$\newline
    $\frac{8}{30}\pm(1.96)\sqrt{\frac{(4/15) \cdot (11/15)}{30}}=\frac{8}{30}\pm 0.158$\newline
    \boldmath$0.108<p<0.425$.
\end{description}

\textbf{Question 9.108:}
A manufacturer turns out a product item that
is labeled either "defective" or "not defective." In order
to estimate the proportion defective, a random sample 
of 100 items is taken from production, and 10 are
found to be defective. Following implementation of a
quality improvement program, the experiment is 
conducted again. A new sample of 100 is taken, and this
time only 6 are found to be defective.
\begin{enumerate}[label = (\alph*) ]
    \item Give a 95\% confidence interval on $p_1 − p_2$ , 
    where $p_1$ is the population proportion defective before 
    improvement and $p_2$ is the proportion defective after
    improvement.
    \item Is there information in the confidence interval
    found in (a) that would suggest that $p_1 > p_2$? 
    Explain.
\end{enumerate}
\begin{description}
    \item \textbf{Answer:} 
\end{description}
\end{document}