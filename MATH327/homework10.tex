\documentclass{article}
\usepackage{amsthm, amsmath, pgfplots, pgfplotstable, enumitem, textcomp}
\usepackage[margin=0.5in]{geometry}
\setlength\parindent{0pt}
\begin{document}
\noindent\textbf{Math 327 Homework 10}\hfill Anchu A. Lee\\
\noindent\today\\\newline
\textbf{Question 10.2:}
A sociologist is concerned about the effectiveness 
of a training course designed to get more drivers
to use seat belts in automobiles.
\begin{enumerate}[label=(\alph*)]
    \item What hypothesis is she testing if she commits a
    type I error by erroneously concluding that the
    training course is ineffective?
    \item What hypothesis is she testing if she commits a
    type II error by erroneously concluding that the
    training course is effective?
\end{enumerate}

\textbf{Question 10.4:}
A fabric manufacturer believes that the proportion 
of orders for raw material arriving late is $p = 0.6$.
If a random sample of 10 orders shows that 3 or fewer
arrived late, the hypothesis that $p = 0.6$ should be
rejected in favor of the alternative $p < 0.6$. Use the
binomial distribution.
\begin{enumerate}[label=(\alph*)]
    \item Find the probability of committing a type I error
    if the true proportion is $p = 0.6$
    \item Find the probability of committing a type II error
    for the alternatives $p = 0.3$, $p = 0.4$, and $p = 0.5$.
\end{enumerate}

\textbf{Question 10.6:}
The proportion of adults living in a small town
who are college graduates is estimated to be $p = 0.6$.
To test this hypothesis, a random sample of 15 adults
is selected. If the number of college graduates in the
sample is anywhere from 6 to 12, we shall not reject
the null hypothesis that $p = 0.6$; otherwise, we shall
conclude that $p \not= 0.6$.
\begin{enumerate}[label=(\alph*)]
    \item Evaluate $\alpha$ assuming that $p = 0.6$. Use the 
    binomial distribution.
    \item Evaluate $\beta$ for the alternatives $p = 0.5$ and 
    $p = 0.7$.
    \item Is this a good test procedure?
\end{enumerate}

\textbf{Question 10.8:}
In \textit{Relief from Arthritis} published by Thorsons
Publishers, Ltd., John E. Croft claims that over 40\%
of those who suffer from osteoarthritis receive 
measurable relief from an ingredient produced by a particular
species of mussel found off the coast of New Zealand.
To test this claim, the mussel extract is to be given to
a group of 7 osteoarthritic patients. If 3 or more of
the patients receive relief, we shall not reject the null
hypothesis that $p = 0.4$; otherwise, we conclude that
$p < 0.4$.
\begin{enumerate}[label=(\alph*)]
    \item Evaluate $\alpha$, assuming that $p = 0.4$.
    \item Evaluate $\beta$ for the alternative $p = 0.3$.
\end{enumerate}

\textbf{Question 10.14:}
A manufacturer has developed a new fishing
line, which the company claims has a mean breaking
strength of 15 kilograms with a standard deviation of
0.5 kilogram. To test the hypothesis that $\mu = 15$ 
kilograms against the alternative that $\mu < 15$ kilograms, a
random sample of 50 lines will be tested. The critical
region is defined to be $\bar{x} < 14.9$.
\begin{enumerate}[label=(\alph*)]
    \item Find the probability of committing a type I error
    when $H_0$ is true.
    \item Evaluate $\beta$ for the alternatives $\mu = 14.8$ and $\mu =
    14.9$ kilograms.
\end{enumerate}

\textbf{Question 10.20:}
A random sample of 64 bags of white cheddar 
popcorn weighed, on average, 5.23 ounces with a
standard deviation of 0.24 ounce. Test the hypothesis
that $\mu = 5.5$ ounces against the alternative hypothesis,
$\mu < 5.5$ ounces, at the 0.05 level of significance.
\newline

\textbf{Question 10.22:}
In the American Heart Association journal 
\textit{Hypertension}, researchers report that individuals who
practice Transcendental Meditation (TM) lower their
blood pressure significantly. If a random sample of 225
male TM practitioners meditate for 8.5 hours per week
with a standard deviation of 2.25 hours, does that suggest 
that, on average, men who use TM meditate more
than 8 hours per week? Quote a $P$-value in your 
conclusion.
\newline

\textbf{Question 10.24:}
The average height of females in the freshman
class of a certain college has historically been 162.5 
centimeters with a standard deviation of 6.9 centimeters.
Is there reason to believe that there has been a change
in the average height if a random sample of 50 females
in the present freshman class has an average height of
165.2 centimeters? Use a $P$-value in your conclusion.
Assume the standard deviation remains the same.
\newline

\textbf{Question 10.26:}
According to a dietary study, high sodium intake 
may be related to ulcers, stomach cancer, and
migraine headaches. The human requirement for salt
is only 220 milligrams per day, which is surpassed in
most single servings of ready-to-eat cereals. If a 
random sample of 20 similar servings of a certain cereal
has a mean sodium content of 244 milligrams and a
standard deviation of 24.5 milligrams, does this 
suggest at the 0.05 level of significance that the average
sodium content for a single serving of such cereal is
greater than 220 milligrams? Assume the distribution
of sodium contents to be normal.
\newline

\textbf{Question 10.30:}
A random sample of size $n_1 = 25$, taken from a
normal population with a standard deviation $\sigma_1 = 5.2$,
has a mean $\bar{x}_1 = 81$. A second random sample of size
$n_2 = 36$, taken from a different normal population with
a standard deviation $\sigma_2 = 3.4$, has a mean $\bar{x}_2 = 76$.
Test the hypothesis that $\mu_1 = \mu_2$ against the alternative, 
$\mu_1 \not= \mu_2$. Quote a $P$-value in your conclusion.
\newline

\textbf{Question 10.32:}
\textit{Amstat News} (December 2004) lists median
salaries for associate professors of statistics at research
institutions and at liberal arts and other institutions
in the United States. Assume that a sample of 200
associate professors from research institutions has an
average salary of \$70,750 per year with a standard 
deviation of \$6000. Assume also that a sample of 200 
associate professors from other types of institutions has
an average salary of \$65,200 with a standard deviation
of \$5000. Test the hypothesis that the mean salary
for associate professors in research institutions is \$2000
higher than for those in other institutions. Use a 0.01
level of significance.
\newline

\textbf{Question 10.36:}
Engineers at a large automobile manufacturing 
company are trying to decide whether to purchase
brand A or brand B tires for the company’s new models. 
To help them arrive at a decision, an experiment
is conducted using 12 of each brand. The tires are run
until they wear out. The results are as follows:
\begin{align*}
    \text{Brand }A \text{: }& \bar{x}_1 = 37,900 \text{ kilometers,}\\
    &s_1 = 5100 \text{ kilometers.}\\
    \text{Brand }B \text{: }& \bar{x}_1 = 39,800 \text{ kilometers,}\\
    &s 2 = 5900 \text{ kilometers.}
\end{align*}
Test the hypothesis that there is no difference in the
average wear of the two brands of tires. Assume the
populations to be approximately normally distributed
with equal variances. Use a $P$-value.
\newline

\textbf{Question 10.40:}
\\
\textbf{Question 10.42:}
\\
\textbf{Question 10.44:}
\\
\textbf{Question 10.52:}
For testing
\begin{align*}
    H_0 &: \mu = 14,\\
    H_1 &: \mu \not= 14,
\end{align*}
an $\alpha = 0.05$ level $t$-test is being considered. What 
sample size is necessary in order for the probability to be
0.1 of falsely failing to reject $H_0$ when the true 
population mean differs from 14 by 0.5? From a preliminary
sample we estimate σ to be 1.25.
\newline

\textbf{Question 10.56:}
Suppose that, in the past, 40\% of all adults
favored capital punishment. Do we have reason to
believe that the proportion of adults favoring capital
punishment has increased if, in a random sample of 15
adults, 8 favor capital punishment? Use a 0.05 level of
significance.
\newline

\textbf{Question 10.58:}
It is believed that at least 60\% of the residents
in a certain area favor an annexation suit by a 
neighboring city. What conclusion would you draw if only
110 in a sample of 200 voters favored the suit? Use a
0.05 level of significance.
\newline

\textbf{Question 10.64:}
In a study on the fertility of married women
conducted by Martin O’Connell and Carolyn C. Rogers
for the Census Bureau in 1979, two groups of childless
wives aged 25 to 29 were selected at random, and each
was asked if she eventually planned to have a child.
One group was selected from among wives married
less than two years and the other from among wives
married five years. Suppose that 240 of the 300 wives
married less than two years planned to have children
some day compared to 288 of the 400 wives married
five years. Can we conclude that the proportion of
wives married less than two years who planned to have
children is significantly higher than the proportion of
wives married five years? Make use of a $P$-value.
\newline

\textbf{Question 10.68:}
Past experience indicates that the time required 
for high school seniors to complete a standardized 
test is a normal random variable with a standard
deviation of 6 minutes. Test the hypothesis that $\sigma = 6$
against the alternative that $\sigma < 6$ if a random sample of
the test times of 20 high school seniors has a standard
deviation $s = 4.51$. Use a 0.05 level of significance.
\newline

\textbf{Question 10.70:}
Past data indicate that the amount of money
contributed by the working residents of a large city to
a volunteer rescue squad is a normal random variable
with a standard deviation of \$1.40. It has been 
suggested that the contributions to the rescue squad from
just the employees of the sanitation department are
much more variable. If the contributions of a random
sample of 12 employees from the sanitation department
have a standard deviation of \$1.75, can we conclude at
the 0.01 level of significance that the standard 
deviation of the contributions of all sanitation workers is
greater than that of all workers living in the city?
\newline

\textbf{Question 10.76:}
\\

\textbf{Question 10.78:}

\end{document}