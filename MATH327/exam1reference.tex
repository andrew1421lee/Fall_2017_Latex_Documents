\documentclass[9pt]{article}
\usepackage{amsthm, amsmath, extsizes}
\usepackage[margin=0.25in]{geometry}
\newcommand*\mean[1]{\bar{#1}}
\begin{document}
    \noindent\textbf{Chapter 1}
    \underline{Sample Mean:} Add all elements and divide by number of elements. 
    \underline{Sample Median:} Middle element of ordered list, or mean of middle two.
    \underline{Trimmed Mean:} Eliminate largest and smallest percentage and then take the mean.
    \underline{Sample Variance:} $s^2 = \sum_{i=1}^{n}=\frac{(x_i-\mean{x})^2}{n-1}$.
    \underline{Degrees of Freedom:} $n-1$.
    \underline{Sample Standard Deviation:} $s = \sqrt{s^2}$.
    \underline{Continuous Data:} Value over a given range, countably infinite.
    \underline{Discrete Data:} Particular value, countable.
    \underline{Scatter Plot:} Just graph that shit.
    \underline{Stem-and-Leaf Plot:} Split values into stems and leaves. Easy way would be the ten's place for the stems, and one's place for the leaves.
    \underline{Reletive Frequency:} Divide the frequency (occurrences) by total number of elements.
    \underline{Histogram:} Mate.
    \underline{Box Plot:} Take the median of the entire data set (2nd quartile), then take the median of the set of data to the left (1st quartile), and the median of the set of data to the right (3rd quartile). Outliers are values outside the range of 1.5 times the IQR.
    \underline{IQR:} The interquartile range is the 3rd quartile minus the 1st quartile.
    \\
    \noindent\textbf{Chapter 2}
    \\
    \noindent\textbf{Chapter 3}
    \\
    \noindent\textbf{Chapter 4}
\end{document}