\documentclass{article}
\usepackage{amsthm, amsmath, pgfplots, enumitem}
\usepackage[margin=0.5in]{geometry}
\begin{document}
    \noindent\textbf{Math 327 Homework 6}\hfill Anchu A. Lee\\
    \noindent\today\\\\
    \noindent\textbf{Question 6.2}: Suppose $X$ follows a continuous uniform distribution 
    from 1 to 5. Determine the conditional probability $P(X>2.5\mid X\leq 4)$
        \begin{center}
            $\frac{P(2.5<X\geq 4)}{P(X\leq 4)} = \frac{4-2.5}{4-1} = \frac{1.5}{3}$
        \end{center}
    \noindent\textbf{Question 6.4}: A bus arrives every 10 minutes at a bus stop. It is 
    assumed that the waiting time for a particular individual is a random 
    variable with a continuous uniform distribution.
        \begin{enumerate}[label = (\alph*) ]
            \item What is the probability that the individual waits 
            more than 7 minutes?
                \begin{center}
                    $P(X>7) = \frac{10-7}{10} = \frac{3}{10}$ 
                \end{center}
            \item  What is the probability that the individual waits
            between 2 and 7 minutes?
                \begin{center}
                    $P(2<X<7) = \frac{7-2}{10} = \frac{5}{10}$
                \end{center}
        \end{enumerate}

    \noindent\textbf{Question 6.6}: Find the value of $z$ if the area under a standard
    normal curve
        \begin{enumerate}[label = (\alph*) ]
            \item to the right of $z$ is $0.3622$;
                \begin{center}
                    The left of $z$ is then $1 - 0.3622 = 0.6378$. $0.6378$ can be approximated as $0.6368$
                    so $z = 0.35$ by Table A.3.
                \end{center}
            \item to the left of $z$ is $0.1131$;
                \begin{center}
                    The right of $z$ is then $0.1131 - 1 = -0.8869$. So by table A.3, $z = -1.21$
                \end{center}
            \item between $0$ and $z$, with $z > 0$, is $0.4838$;
                \begin{center}
                    Area left of $z$ is $0.5 + 0.4838 = 0.9838$. So $z = 2.14$.
                \end{center}
            \item between $−z$ and $z$, with $z > 0$, is $0.9500$.
                \begin{center}
                    Area left of $z$ is $0.25 + 0.95 = 0.975$, so $z = 1.96$.
                \end{center}
        \end{enumerate}

    \noindent\textbf{Question 6.10}: According to Chebyshev’s theorem, the probability 
    that any random variable assumes a value within 3 standard deviations of 
    the mean is at least $8/9$. If it is known that the probability distribution 
    of a random variable $X$ is normal with mean $\mu$ and variance $\sigma^2$,
    what is the exact value of $P(\mu - 3\sigma < X < \mu + 3\sigma)$?
        \begin{align*}
            z_1 = \frac{((\mu - 3\sigma) - \mu)}{\sigma} &= -3\\
            z_2 = \frac{((\mu + 3\sigma) - \mu)}{\sigma} &= 3 &&\text{So then:}\\ 
            P(\mu - 3\sigma < X < \mu + 3\sigma) &= P(-3 < Z < 3)\\
                                                 &= 0.9987 - 0.0013 &&\text{by Table A.3}\\
                                                 &= 0.9974
        \end{align*}
    \textbf{Question 6.12}: The loaves of rye bread distributed to local
    stores by a certain bakery have an average length of 30 centimeters 
    and a standard deviation of 2 centimeters. Assuming that the lengths 
    are normally distributed, what percentage of the loaves are
        \begin{enumerate}[label = (\alph*) ]
            \item longer than 31.7 centimeters?
                \begin{align*}
                    z = \frac{31.7 - 30}{2} &= 0.85\\
                    P(X > 31.7) = P(Z > 0.85) &= 0.1977 &&\text{by Table A.3}
                \end{align*}
                19.77\% of loaves are longer than 31.7 cm.
            \item between 29.3 and 33.5 centimeters in length?
                \begin{align*}
                    z_1 = \frac{29.3-30}{2} &= -0.35\\
                    z_2 = \frac{33.5-30}{2} &= 1.75\\
                    P(29.3 < X < 33.5) &= P(-0.35 < Z < 1.75)\\
                                       &= 0.9599 - 0.3632 &&\text{by Table A.3}\\
                                       &= 0.5967
                \end{align*}
                59.67\% of loaves are within 29.3 and 33.5 cm.
            \item shorter than 25.5 centimeters?
                \begin{align*}
                    z = \frac{25.5-30}{2} &= -2.25\\
                    P(X < 25.5) &= P(Z < 2.25)\\
                                &= 0.0122 &&\text{by Table A.3}
                \end{align*}
                1.22\% of loaves are shorter than 25.5 cm.
        \end{enumerate}

    \noindent\textbf{Question 6.22}: If a set of observations is normally distributed,
    what percent of these differ from the mean by
        \begin{enumerate}[label = (\alph*) ]
            \item more than $1.3\sigma$?
                \begin{align*}
                    x_1 &= \mu + 1.3\sigma\\
                    x_2 &= \mu - 1.3\sigma &&\text{Therefore:}\\
                    z_1 &= 1.3\\
                    z_2 &= -1.3\\
                    P(X>\mu + 1.3\sigma)+P(X < 1.3\sigma) &= P(Z>1.3) + P(Z<-1.3)\\
                                                          &= 0.0968 + 0.0968 &&\text{by Table A.3}\\
                                                          &= 0.1936
                \end{align*}
                19.36\% differ from the mean by more than $1.3\sigma$.
            \item less than $0.52\sigma$?
                \begin{align*}
                    x_1 = \mu+0.52\sigma && z_1 &= 0.52\\
                    x_2 = \mu-0.52\sigma && z_2 &= -0.52\\
                    P(\mu - 0.52\sigma < X < \mu + 0.52\sigma) &= P(-0.52 < Z < 0.52) \\
                                                               &= 0.6985 - 0.3015 &&\text{by Table A.3}\\
                                                               &= 0.3970
                \end{align*}
                39.70\% differ from the mean by less than $0.52\sigma$
        \end{enumerate}

    \noindent\textbf{Question 6.26}: A process yields 10\% defective items. If 100
    items are randomly selected from the process, what is the probability 
    that the number of defectives
        \begin{enumerate}[label = (\alph*) ]
            \item exceeds 13?
                \begin{align*}
                    \mu = np = (100)(0.1) &= 10\\
                    \sigma = \sqrt{(100)(0.1)(0.9)} &= 3\\
                    z = \frac{13.5-10}{3} &= 1.17\\
                    P(X>13.5) &= P(Z>1.17) &&\text{by Table A.3}\\
                    &= 0.1210
                \end{align*}
            12.10\% number of defects exceeds 13.
            \item is less than 8?
                \begin{align*}
                    z = \frac{7.5 - 10}{3} &= -0.83\\
                    P(X < 7.5) &= P(Z < -0.83)&&\text{by Table A.3} \\
                    &= 0.2033
                \end{align*}
            20.33\% number of defects is below 8.
        \end{enumerate}

    \noindent\textbf{Question 6.30}: A drug manufacturer claims that a certain drug
    cures a blood disease, on the average, 80\% of the time. To check the 
    claim, government testers use the drug on a sample of 100 individuals 
    and decide to accept the claim if 75 or more are cured.
        \begin{enumerate}[label = (\alph*) ]
            \item What is the probability that the claim will be rejected 
            when the cure probability is, in fact, 0.8?
                \begin{align*}
                    \mu = (100)(0.8) &= 80\\
                    \sigma = \sqrt{(100)(0.8)(0.2)} &= 4\\
                    z = \frac{74.5-80}{4} &= -1.38 &&\text{by Table A.3}\\
                    &= 0.0838 && (8.38\%)
                \end{align*}
            \item What is the probability that the claim will be accepted 
            by the government when the cure probability is as low as 0.7?
                \begin{align*}
                    \mu = (100)(0.7) &= 70\\
                    \sigma = \sqrt{(100)(0.7)(0.3)} &= 4.583\\
                    z = \frac{74.5-70}{4.583} &= 0.98 &&\text{by Table A.3}\\
                    &= 1 - 0.8365 = 0.1635 && (16.35\%)
                \end{align*}
        \end{enumerate}

    \noindent\textbf{Question 6.34}: A pair of dice is rolled 180 times. What is the
    probability that a total of 7 occurs
        \begin{enumerate}[label = (\alph*) ]
            \item at least 25 times?
                \begin{align*}
                    \mu = (180)(1/6) &= 30\\
                    \sigma = \sqrt{(180)(1/2)(5/6)} &= 5\\
                    z = \frac{24.5-30}{5} &= -1.1\\
                    P(X > 24.5) &= P(Z > -1.1) &&\text{by Table A.3}\\
                    &= 1-0.1357 = 0.8643 && (86.43\%)
                \end{align*}
            \item between 33 and 41 times inclusive?
                \begin{align*}
                    z_1=\frac{32.5-30}{5} &= 0.5\\
                    z_2=\frac{41.5-30}{5} &= 2.3\\
                    P(32.5 < X < 41.5) &= P(0.5 < Z < 2.3) &&\text{by Table A.3}\\
                    &= 0.9893 - 0.6915 = 0.2978 &&(29.78\%)
                \end{align*}
            \item exactly 30 times?
                \begin{align*}
                    z_1 = \frac{29.5-30}{5} &= -0.1\\
                    z_2 = \frac{30.5-30}{5} &= 0.1\\
                    P(29.5 < X < 30.5) &= P(-0.1 < Z < 0.1) &&\text{by Table A.3}\\
                    &= 0.5398 - 0.4602 = 0.0796 && (7.96\%)
                \end{align*}
        \end{enumerate}

    \noindent\textbf{Question 6.40}: In a certain city, the daily consumption of water
    (in millions of liters) follows approximately a gamma distribution with 
    $\alpha = 2$ and $\beta = 3$. If the daily capacity of that city is 9 
    million liters of water, what is the probability that on any given day 
    the water supply is inadequate?
        \begin{align*}
            P(X>9) &= \frac{1}{9}\int_9^\infty xe^{-x/3}dx\\
            &\approx 0.19915
        \end{align*}

    \noindent\textbf{Question 6.42}: Suppose that the time, in hours, required to
    repair a heat pump is a random variable $X$ having a gamma distribution 
    with parameters $\alpha = 2$ and $\beta = 1/2$. What is the probability 
    that on the next service call
        \begin{enumerate}[label = (\alph*) ]
            \item at most 1 hour will be required to repair the heat pump?
                \begin{align*}
                    P(X<1) &= 4\int_0^1 xe^{-2x}dx\\
                    &\approx 0.5940
                \end{align*}
            \item at least 2 hours will be required to repair the heat pump?
                \begin{align*}
                    P(X>2) &= 4\int_2^\infty xe^{-2x}dx\\
                    &\approx 0.0916
                \end{align*}
        \end{enumerate}

    \noindent\textbf{Question 6.46}: The life, in years, of a certain type of electrical
    switch has an exponential distribution with an average life $\beta = 2$. If 100 of 
    these switches are installed in different systems, what is the probability that at 
    most 30 fail during the first year?
        \begin{align*}
            P(X<1) = \frac{1}{2}\int_0^1 e^{-x/2}dx &\approx 0.39347\\
            \mu = (100)(0.39347) &= 39.347\\
            \sigma = \sqrt{(100)(0.39347)(0.6065)} &= 4.885\\
            z = \frac{30.5-39.347}{4.885} &= -1.81\\
            P(X\leq 30) = P(Z < -1.81) &= 0.0352 &&\text{by Table A.3}
        \end{align*}
    
    \noindent\textbf{Question 6.50}: If the proportion of a brand of television set 
    requiring service during the first year of operation is a random variable having 
    a beta distribution with $\alpha = 3$ and $\beta = 2$, what is the probability 
    that at least 80\% of the new models of this brand sold this year will require
    service during their first year of operation?\\\\
    
    \noindent\textbf{Question 6.54}: The lifetime, in weeks, of a certain type of 
    transistor is known to follow a gamma distribution with mean 10 weeks and standard 
    deviation $\sqrt50$ weeks.
        \begin{enumerate}[label = (\alph*) ]
            \item What is the probability that a transistor of this type will last at 
            most 50 weeks?
            \item  What is the probability that a transistor of this type will not 
            survive the first 10 weeks?
        \end{enumerate}
    
    \noindent\textbf{Question 6.56}: Rate data often follow a lognormal distribution.
    Average power usage (dB per hour) for a particular company is studied and is known 
    to have a lognormal distribution with parameters $\mu = 4$ and $\sigma = 2$. What
    is the probability that the company uses more than 270 dB during any particular hour?\\\\
    
    \noindent\textbf{Question 6.58}: The number of automobiles that arrive at a certain 
    intersection per minute has a Poisson distribution with a mean of 5. Interest centers 
    around the time that elapses before 10 automobiles appear at the intersection.
        \begin{enumerate}[label = (\alph*) ]
            \item What is the probability that more than 10 automobiles appear at the 
            intersection during any given minute of time?
            \item What is the probability that more than 2 minutes elapse before 10 
            cars arrive?
        \end{enumerate}
\end{document}