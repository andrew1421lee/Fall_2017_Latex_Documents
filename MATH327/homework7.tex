\documentclass{article}
\usepackage{amsthm, amsmath, pgfplots, enumitem}
\usepackage[margin=0.5in]{geometry}
\begin{document}
    \noindent\textbf{Math 327 Homework 6}\hfill Anchu A. Lee\\
    \noindent\today\\\\
    \noindent\textbf{Question 8.2}: The lengths of time, in minutes, that 10 patients
    waited in a doctor’s office before receiving treatment
    were recorded as follows: 5, 11, 9, 5, 10, 15, 6, 10, 5,
    and 10. Treating the data as a random sample, find
        \begin{enumerate}[label = (\alph*) ]
            \item the mean;\newline
            \textbf{8.6 minutes}
            \item the median;\newline
            \textbf{9.5 minues}
            \item the mode.\newline
            \textbf{5 and 10 minutes}
        \end{enumerate}
    \textbf{Question 8.6}: Find the mean, median, and mode for the sample
    whose observations, 15, 7, 8, 95, 19, 12, 8, 22, and 14,
    represent the number of sick days claimed on 9 fed-
    eral income tax returns. Which value appears to be
    the best measure of the center of these data? State
    reasons for your preference.\\\newline
    \textbf{Mean: 22.222...; Median: 14; Mode: 8\newline The median is the best
    mesure of the center for this data. Mean is skewed because of one outlier
    and mode doesn't reflect the center.}\\\newline
    \textbf{Question 8.12}: The tar contents of 8 brands of cigarettes se-
    lected at random from the latest list released by the
    Federal Trade Commission are as follows: 7.3, 8.6, 10.4,
    16.1, 12.2, 15.1, 14.5, and 9.3 milligrams. Calculate
    7.3 8.6 9.3 10.4 12.2 14.5 15.1 16.1
        \begin{enumerate}[label = (\alph*) ]
            \item the mean;\newline
            \textbf{11.69 milligrams}
            \item the variance.\newline
            \textbf{10.77 milligrams}
        \end{enumerate}
    \textbf{Question 8.18}: If the standard deviation of the mean for the
    sampling distribution of random samples of size 36
    from a large or infinite population is 2, how large must
    the sample size become if the standard deviation is to
    be reduced to 1.2?\\\newline
        \textbf{$4=variance/(36)$, so variance is 144. Then $1.44=144/x$;
        so the sample size must be at least 100 to have a standard deviation of 
        1.2.}\\\newline
    \textbf{Question 8.22}: The heights of 1000 students are approximately
    normally distributed with a mean of 174.5 centimeters
    and a standard deviation of 6.9 centimeters. Suppose
    200 random samples of size 25 are drawn from this pop-
    ulation and the means recorded to the nearest tenth of
    a centimeter. Determine
        \begin{enumerate}[label = (\alph*) ]
            \item the mean and standard deviation of the sampling
            distribution of $\bar{X}$;\newline
            \textbf{Mean: 174.5cm; Standard deviation: $6.9/\sqrt{25} = 1.38$}.
            \item the number of sample means that fall between 172.5
            and 175.8 centimeters inclusive;\newline
            \textbf{$P(172.45 < \bar{X} < 175.85)$, as standard deviation is 1.38;
            $z_1 = (172.45-174.5)/1.38 = -1.49$, $z_2 = (175.85-174.5)/1.38 = 0.98$.
            then $P(-1.49 < Z < 0.98)$ which according to table A.3 is $0.8365-0.0681$
             and the probability is 0.7684 and the number of samples is 154.}
            \item the number of sample means falling below 172.0
            centimeters.\newline
            \textbf{$P(\bar{X} < 171.95)$, then $z = (171.95-174.5)/1.38 = -1.85$
            then $P(Z < -1.85)$ and by table A.3 equals $0.0322$ and the number of 
             samples is 6.}
        \end{enumerate}
    \textbf{Question 8.24}: If a certain machine makes electrical resistors
    having a mean resistance of 40 ohms and a standard
    deviation of 2 ohms, what is the probability that a
    random sample of 36 of these resistors will have a com-
    bined resistance of more than 1458 ohms?\\\newline
    \textbf{$P(\sum_{i=1}^{36}X_i > 1458) = P(\bar{X} > 1458/36)$ now find for Z:
    $ \sigma = 2/\sqrt{36} = 1/3$ and $z=(40.5 - 40)/1.5 = 1.5$ so now $P(Z > 1.5)$
    which according to table A.3 is $1-0.9332$ and the answer is 0.0668}\\\newline
    \textbf{Question 8.28}: A random sample of size 25 is taken from a 
    normal population having a mean of 80 and a standard
    deviation of 5. A second random sample of size 36
    is taken from a different normal population having a
    mean of 75 and a standard deviation of 3. Find the
    probability that the sample mean computed from the
    25 measurements will exceed the sample mean computed 
    from the 36 measurements by at least 3.4 but
    less than 5.9. Assume the difference of the means to
    be measured to the nearest tenth.\\\newline
    \textbf{Need to find $P(3.35 < \bar{X}_1 - \bar{X}_2 < 5.85)$. The mean 
    difference would be $\mu_1 - \mu_2 = 80 - 75 = 5$. and the standard 
    deviation difference would be $\sqrt{1 + 1/4} = 1.12$. Then $z_1=(3.35-5)/1.12=-1.47$
    and $z_2=(5.85-5)/1.12 = 0.76$. Now $P(-1.47< Z < 0.76) = 0.7764-0.0708$ so the
    answer is 0.7056}\\\newline
    \textbf{Question 8.32}: Two different box-filling machines are used to fill
    cereal boxes on an assembly line. The critical measure-
    ment influenced by these machines is the weight of the
    product in the boxes. Engineers are quite certain that
    the variance of the weight of product is $\sigma^2 = 1$ ounce.
    Experiments are conducted using both machines with
    sample sizes of 36 each. The sample averages for ma-
    chines $A$ and $B$ are $\bar{x}_A = 4.5$ ounces and $\bar{x}_B = 4.7$
    ounces. Engineers are surprised that the two sample
    averages for the filling machines are so different.
        \begin{enumerate}[label = (\alph*) ]
            \item Use the Central Limit Theorem to determine $P(\bar{X}_B - \bar{X}_A \geq 0.2)$
            under the condition that $\mu_A = \mu_B$.\newline
                \textbf{$P(\bar{X}_B-\bar{X}_A \geq 0.2)$. $z=0.2/\sqrt{\frac{1}{36}+\frac{1}{36}} = 0.85$
                Then $P(Z \geq 0.85) = 1-0.8023 =$ 0.1977}
            \item Do the aforementioned experiments seem to, in any
            way, strongly support a conjecture that the popu-
            lation means for the two machines are different?
            Explain using your answer in (a).\newline
                \textbf{No, it does not support the conjecture. 
                If the two population means were different, then
                the results from (a) would be smaller.}
        \end{enumerate}
    \textbf{Question 8.34}: Two alloys $A$ and $B$ are being used to manufac-
    ture a certain steel product. An experiment needs to
    be designed to compare the two in terms of maximum
    load capacity in tons (the maximum weight that can
    be tolerated without breaking). It is known that the
    two standard deviations in load capacity are equal at
    5 tons each. An experiment is conducted in which 30
    specimens of each alloy ($A$ and $B$) are tested and the
    results recorded as follows:\newline
    :\hfill $\bar{x}_A = 49.5$, \hfill $\bar{x}_B=45.5$; \hfill $\bar{x}_A-\bar{x}_B = 4$ \hfill.\newline
    The manufacturers of alloy A are convinced that this
    evidence shows conclusively that $\mu_A > \mu_B$ and strongly
    supports the claim that their alloy is superior. Man-
    ufacturers of alloy B claim that the experiment could
    easily have given $\bar{x}_A - \bar{x}_B = 4$ even if the two popula-
    tion means are equal. In other words, “the results are
    inconclusive!”
    \begin{enumerate}[label = (\alph*) ]
        \item Make an argument that manufacturers of alloy B
        are wrong. Do it by computing $P(\bar{X}_A-\bar{X}_B>4 \mid \mu_A = \mu_B)$.\newline
            \textbf{$\sqrt{5^2/30 + 5^2/30} = 1.29$ so then $(4-0)/1.29=3.19$. 
            $P(\bar{X}_A-\bar{X}_B>4 \mid \mu_A = \mu_B) = P(Z>3.10)=$ 0.0010}
        \item Do you think these data strongly support alloy $A$?\newline
            \textbf{As the answer from (a) is very miniscule, it supports alloy A's statement.}
    \end{enumerate}
    \textbf{Question 8.38}: For a chi-squared distribution, find
        \begin{enumerate}[label = (\alph*) ]
            \item $x_{0.005}^{2}$ when $v = 5$;\newline
                \textbf{27.49}
            \item $x_{0.05}^{2}$ when $v = 19$;\newline
                \textbf{18.48}
            \item $x_{0.01}^{2}$ when $v = 12$.\newline
                \textbf{36.42}
        \end{enumerate}
    \textbf{Question 8.40}: For a chi-squared distribution, find $x^{2}_{\alpha}$ such that
        \begin{enumerate}[label = (\alph*) ]
            \item $P(X^2 > x^{2}_{\alpha}) = 0.01$ when $v = 21$;\newline
                \textbf{$X_{0.01}^{2} = $ 38.93}
            \item $P(X^2 < x^{2}_{\alpha}) = 0.95$ when $v = 6$;\newline
                \textbf{$X_{0.05}^{2} = $ 12.59}
            \item $P(x^{2}_{\alpha} < X^2 < 23.209) = 0.015$ when v = 10.\newline
                \textbf{$X_{0.01}^{2}= $ 23.209 and $0.01 + 0.015 = 0.025$ so $X_{0.025}^{2}= $ 20.483}
        \end{enumerate}
    \textbf{Question 8.42}: The scores on a placement test given to college
    freshmen for the past five years are approximately nor-
    mally distributed with a mean $\mu = 74$ and a variance
    $\sigma^2 = 8$. Would you still consider $\sigma^2 = 8$ to be a valid
    value of the variance if a random sample of 20 students
    who take the placement test this year obtain a value of
    $s^2 = 20$?\\\newline
        \textbf{$X^2 = \frac{19\cdot20}{8} = 47.5$. Meanwhile $X_{0.01}^{2} = 36.191$
        so the value of variance is not valid.}\\\newline
    \textbf{Question 8.44}:
        \begin{enumerate}[label = (\alph*) ]
            \item Find $t_{0.023}$ when $v = 14$.
            \item Find $-t_{0.10}$ when $v = 10$.
            \item find $t_0.995$ when $v = 7$.
        \end{enumerate}
    \textbf{Question 8.46}:
        \begin{enumerate}[label = (\alph*) ]
            \item Find $P(-t_{0.005} < T < t_{0.01})$ for $v = 20$.
            \item Find $P(T > -t_{0.025})$.
        \end{enumerate}
    \textbf{Question 8.48}: A manufacturing firm claims that the batteries
    used in their electronic games will last an average of
    30 hours. To maintain this average, 16 batteries are
    tested each month. If the computed t-value falls be-
    tween $-t_{0.025}$ and $t_{0.025}$ , the firm is satisfied with its
    claim. What conclusion should the firm draw from a
    sample that has a mean of $\bar{x} = 27.5$ hours and a stan-
    dard deviation of $s = 5$ hours? Assume the distribution
    of battery lives to be approximately normal.\\\newline
    \textbf{Question 8.50}: A maker of a certain brand of low-fat cereal bars
    claims that the average saturated fat content is 0.5
    gram. In a random sample of 8 cereal bars of this
    brand, the saturated fat content was 0.6, 0.7, 0.7, 0.3,
    0.4, 0.5, 0.4, and 0.2. Would you agree with the claim?
    Assume a normal distribution.\\\newline
    \textbf{Question 8.51}: For an $F$-distribution, find
        \begin{enumerate}[label = (\alph*) ]
            \item $f_{0.05}$ with $v_1 = 7$ and $v_2 = 15$;
            \item $f_{0.05}$ with $v_1 = 15$ and $v_2 = 7$:
            \item $f_{0.01}$ with $v_1 = 24$ and $v_2 = 19$;
            \item $f_{0.95}$ with $v_1 = 19$ and $v_2 = 24$;
            \item $f_{0.99}$ with $v_1 = 28$ and $v_2 = 12$.
        \end{enumerate}
    \textbf{Question 8.52}: Pull-strength tests on 10 soldered leads for a
    semiconductor device yield the following results, in
    pounds of force required to rupture the bond:
        \begin{center}
            \begin{tabular}{c c c c c}
                19.8 & 12.7 & 13.2 & 16.9 & 10.6 \\
                18.8 & 11.1 & 14.3 & 17.0 & 12.5
            \end{tabular}
        \end{center}
    Another set of 8 leads was tested after encapsulation
    to determine whether the pull strength had been in-
    creased by encapsulation of the device, with the fol-
    lowing results:
        \begin{center}
            \begin{tabular}{c c c c c c c c}
                24.9 & 22.8 & 23.6 & 22.1 & 20.4 & 21.6 & 21.8 & 22.5
            \end{tabular}
        \end{center}
    Comment on the evidence available concerning equal-
    ity of the two population variances.\\\newline
    \textbf{Question 8.54}: Construct a quantile plot of these data, which
    represent the lifetimes, in hours, of fifty 40-watt, 110-
    volt internally frosted incandescent lamps taken from
    forced life tests:
        \begin{center}
            \begin{tabular}{c c c c c c}
                919 & 1196 & 785 & 1126 & 936 & 918 \\
                1156 & 920 & 948 & 1067 & 1092 & 1162 \\
                1170 & 929 & 950 & 905 & 972 & 1035 \\
                1045 & 855 & 1195 & 1195 & 1340 & 1122 \\
                938 & 970 & 1237 & 956  & 1102 & 1157 \\
                978 & 832 & 1009 & 1157 & 1151 & 1009 \\
                765 & 958 & 902 & 1022 & 1333 & 811 \\ 
                1217 & 1085 & 896 & 958 & 1311 & 1037 \\
                702 & 923
            \end{tabular}
        \end{center}
    \textbf{Question 8.64}: If $S^2_1$ and $S_2^2$ represent the variances of indepen-
    dent random samples of size $n_1 = 25$ and $n_2 = 31$,
    taken from normal populations with variances $\sigma^2_1 = 10$
    and $\sigma_2^2 = 15$, respectively, find $P(\sigma^2_1 / \sigma_2^2 > 1.26)$.
\end{document}