\documentclass[9pt]{article}
\usepackage{amsthm, amsmath, extsizes, multicol, amsfonts, enumitem, tabularx, wrapfig}
\usepackage[normalem]{ulem}
\usepackage[margin=0.25in]{geometry}
\newcommand*\mean[1]{\bar{#1}}
\newcommand*\median[1]{\tilde{#1}}

\setlength{\multicolsep}{2pt}
\begin{document}
\addtolength{\tabcolsep}{-2pt}
\begin{multicols*}{2}
    \noindent\textbf{Midterm 1: Chapters 1 to 4}
    
    \noindent\begin{tabular*}{0.5\textwidth}{c c c c c c c c c c}
        \hline
        1 & 2 & 3 & 4 & 5 & 6 & 7 & 8 & 9 & 10\\
        22.43 & 10.25 & 23.71 & 21.77 & 22.11 & 18.71 & 19.77 & 20.33 & 20.17 & 21.12\\
        \hline
    \end{tabular*}
    (a) $\mean{x}=\frac{\Sigma_{i=1}^{10}x_i}{10} = 20.037$. 
    $\median{x}=\frac{20.33 + 21.12}{2} = 20.725$\newline
    (b) $s_{x}^{2}=\frac{\Sigma(x_i-\mean{x})^2}{n-1} = 13.935$. 
    Split into 4 sections, numbers seperating are quartiles. 
    Last minus first is IQR $=22.11-19.77 = 2.34$\newline
    (c) Trimmed mean of 10\%: remove 10\% from highest and lowest. $= 20.801$.
    Close to median but more than mean; data is slightly skewed to the left.\newline
    (d) Set decimal point to $\mid$.\newline
    (e) Away from Q1 and Q3 by $1.5\cdot$IQR are outliers. Left dot is minimum, start of
    box is Q1, middle line is median, end of box is Q3, last dot is maximum.
    \rule{0.5\textwidth}{0.4pt}
    \noindent A: polluted, B: test detects pollution, $P(A)=0.2$, $P(B\mid A)=0.60$, 
    $P(B\mid A')=0.3$\newline
    (a) $P(A\cap B)=P(A)P(B\mid A) = 0.2\cdot 0.6=0.12$\newline
    (b) $P(B) = P(B\cap A) + P(B\cap A') = 0.12 + P(A')P(B\mid A') = 0.12 + 0.8 
    \cdot 0.3 = 0.36$\newline
    (c) $P(A\cup B) = P(A)+P(B) - P(A\cap B) = 0.2 + 0.36 - 0.12 = 0.44$\newline
    (d) $P(A'\mid B')=\frac{P(A'\cap B')}{P(B')} = \frac{1 - 0.44}{0.64} = 0.875$
    (e) $P(P(A\cap B) \not= 0)$ so are not mutually exclusive.
    $P(A\cap B) \not= P(A)P(B)$ so are not independent.
    \rule{0.5\textwidth}{0.4pt}
    \noindent $X$ number of cash registers being used for location 1, 
    $Y$ the number used at the same time for location 2.
    \begin{wraptable}{l}{3.8cm}
        \begin{tabular}{c | c | c c c}
            \multicolumn{2}{c|}{} & \multicolumn{3}{c}{$y$}\\
            \multicolumn{2}{c|}{$f(x,y)$} & 0 & 1 & 2\\
            \hline
            & 0 & 0.10 & 0.05 & 0.05\\
            $x$ & 1 & 0.10 & 0.20 & 0.05\\
            & 2 & 0.05 & 0.10 & 0.30\\
            $h(y)$ & & 0.25 & 0.35 & 0.40
        \end{tabular}
    \end{wraptable}
    (a) Marginal probability mass functions, add the rows for $X$; columns for $Y$. $h(y)$ example. \newline
    (b) Cumulative distribution function of $X$: $0$ if $x < 0$, $0.2$ if $0\leq x < 1$.
    $0.55$ if $1\leq x < 2$. $1$ if $x \geq 2$. So $F(1.5) = 0.55$.\newline
    (c) Conditional distribution of $Y$ given $X=2$, $f(y\mid X=2)$: $\frac{0.05}{0.45}$ 
    when $y=0$, $\frac{0.10}{0.45}$ when $y=1$, $\frac{0.30}{0.45}$ when $y=2$.\newline
    (d) Mean of $X$: $\mu_X=E(x) = \Sigma_{x=0}^{2}xg(x) = 0.35 + 2\cdot 0.45 = 1.25$
    Variance of $X$: $\sigma^{2}_{X}=E(X^2)-\mu_{X}^{2} = \Sigma_{x=0}^{2}x^2g(x)-\mu_{X}^{2}
    = 0.35 + 4 \cdot 0.45 - 1.25^2 = 0.5875$\newline
    (e) $\sigma_{XY}=E(XY)-\mu_X\mu_Y = (0.20\cdot 1\cdot 1 + 0.10 \cdot 2\cdot 1 + 0.05 
    \cdot 1 \cdot 2 + 0.30 \cdot 2 \cdot 2) - 1.25\cdot 1.15 = 0.2625 \not=0$ so $X$ and $Y$ are
    not independent.
    \rule{0.5\textwidth}{0.4pt}
    \noindent $X$ be a continuous random variable with probability density function $f(x)=Cx^2$ if 
    $-2<x<1$ and zero otherwise.\newline
    (a) Find $C$, it must make the function equal to 1 for the interval. 
    $1=\int_{-2}^{1}Cx^2dx$, $C = \frac{1}{3}$\newline
    (b) $\mu = E[X] = \int_{-2}^{1}x(\frac{1}{3}x^2)dx = -\frac{5}{4}$. $\sigma^2 = E[X^2]-\mu^2 = 
    \int_{-2}^{1}x^2(\frac{1}{3}x^2)dx - (-\frac{5}{4})^2 = 0.6375$\newline
    (c) $P[X < -1] = P(-2 < x < -1) = \int_{-2}^{-1}\frac{1}{3}x^2dx = \frac{7}{9}$\newline
    (d) $P[-1 < X \leq 3] = P(-1 < X < 1) + P(1 < X < 3) = 
    \int_{-1}^{1}\frac{1}{3}x^2 + \int_{1}^{3}\frac{1}{3}x^2 = \frac{2}{9} + 0 = \frac{2}{9}$\newline
    (e) $g(X) = 4X-3$. $\mu_{g(X)} = \int_{-2}^{1}g(x)f(x)dx = \int_{-2}^{1}(4X-3)(\frac{1}{3}x^2) = -8$. $ 
    \sigma_{g(X)}^{2}=E(g(X)^2)-\mu_{g(X)}^2 = \int_{-2}^{1}(4X-3)^2(\frac{1}{3}x^2)-64 = 10.2$
    \rule{0.5\textwidth}{0.4pt}
    \uline{Combinations} Different order is still same set. $\binom{n}{r} = \frac{n!}{r!(n-r)!}$
    \uline{Permutations} Different order is different set. ($_nP_r$) $=\frac{n!}{(n-r)!}$\newline
    \uline{Chebyshev's Theorem} The probability that a random variable $X$ will be within $k$ standard deviations of the mean is $P(\mu - k\sigma<X<\mu +k\sigma)\geq 1 - \frac{1}{k^2y}$.\newline
    \noindent\textbf{Midterm 2: Chapters 5 to 9}\newline
    There are on average, 3 potholes in a section of 1 mile. Over time, so use Poisson Distribution $\frac{e^{-\lambda t}(\lambda t)^x}{x!}$, $\lambda = 3$\newline
    (a) Probability at least 2 potholes appear in a section of 1 mile. 
    $P(X<2) = P(X=0) + P(X=1) = \frac{e^{-3}3^{0}}{0!} + \frac{e^{-3}3^{1}}{1!} = 0.20$\newline
    (b) Probability at least 2 but less than 4 appear in a section of 1 mile.
    $P(2 \leq X < 4) = P(X=2) + P(X=3) = \frac{e^{-3}3^{2}}{2!} + \frac{e^{-3}3^{3}}{3!} = 0.45$\newline
    (c) Exactly 5 potholes occur in a section of 2 miles.
    $P(X=5) = \frac{e^{-(3\cdot 2)}\cdot (3\cdot 2)^5}{5!} = 0.161$\newline
    (d) Probability of having less than 2 potholes in exactly 2 of the next 5 miles. Binomial, either succeed or fail.
    $b(2; 5, 0.2) = \binom{5}{2}\cdot 0.2^2\cdot 0.8^3 = 0.20$\newline
    (e) Probability the first section of 1 mile with less than 2 potholes is the 4th mile. Geometric, xth trial is the first success.
    $g(4, 0.20) = (0.20)(0.8)^{4-1} = 0.1024$\newline
    \rule{0.5\textwidth}{0.4pt}
    Life of a electrical switch has an exponential distribution with a mean of 1.45 ($\beta = 1.45$) and parameter $\beta$ given by $f(x;\beta) = \frac{1}{\beta}e^{-x/\beta}$ when $x>0$ otherwise 0. $\beta$ is the mean time between failures.\newline
    (a) Probability a randomly selected switch fails within a year.
    $P(X<1) = 1 - e^{(-1/1.45) \cdot 1} = 0.50$\newline
    (b) Switch functions more than a year but less than 2.9 years.
    $P(1 < X < 2.9) = \int_{1}^{2.9}\frac{1}{1.45}e^{-x/1.45} = 0.37$ \newline
    (c) Switch was functioning for 2 years, probability that it will function for another 3.48 years.
    $P(x > 3.48 + 2 \mid X>2) = P(X > 3.48) $ memoryless $ = e^{-3.48/1.45}=0.09$\newline
    (d) If 10 switches are installed what is the probability exactly 3 fail during the first year?
    $Y \sim b(10, 0.5 )$, $P(Y=3) = \binom{10}{3}0.5^3\cdot 0.5^7 = 0.117$\newline
    (e) If 100 switches are installed what is the probability that at least 40 but less than 55 fail within the first year?
    $Y\sim b(100, 0.5)$, $\mu_Y = n\cdot p = 50$, $\sigma_Y = n\cdot p\cdot q = 25$. Because $np=nq = 50 >\sqrt{npq} = 5 $, use Normal distribution to approximate. 
    $Y\approx Normal(50,5)$. $P(40 \leq Y < 55) = P(39.5 \leq Y < 54.5) $ to standard normal $Z=\frac{X-\mu}{\sigma}$,  $= P(\frac{39.5-50}{5} \leq \frac{Y-50}{5} < \frac{54.5-50}{5})
    = P(-2.1 \leq Z < 0.9) $ Check table values. $= 0.8159 - 0.0179 = 0.798$.
    \rule{0.5\textwidth}{0.4pt}
    Approximately normally distributed with a mean of 175cm and a standard deviation of 6cm. $X\sim N(175, 6)$\newline
    (a) Probability that a randomly selected student is at least 170 cm.
    $P(X\geq 170) = P(\frac{X-175}{6} \geq \frac{170-175}{6}) = P(Z\geq -0.83) = 1 - 0.2033 = 0.7967$\newline
    (b) Probability the student's height is between 170 and 180.
    $P(170 < X < 180) = P(-0.83 < Z < 0.83) = 1 - P(Z < -0.83) - P(Z > 0.83) = 1 - 0.2033 - (1 - 0.7967) = 0.59$\newline
    (c) Probability sample of 16 students have an average height is less than 170.
    Using CLT, $Z = \frac{\mean{X}-\mu}{\sigma/\sqrt{n}}$ is a standard normal distribution.
    $\frac{170 - 175}{6/\sqrt{16}} = -3.3\bar{3}$. $P(Z < -3.3\bar{3}) = 0.0004$\newline
    (d) Continuing from (c), would the probability the average height is between 170 and 180 be bigger, smaller, or the same as (b)?
    The larger the sample size, the smaller the variance of sample mean. So it will be bigger.\newline
    (e) Calcualte a 95\% confidence interval of the mean height. $n=25$, $\mean{x} = 170$, $s = 5$. $\mu$ known, $\sigma^2$ unknown.
    CI = $\mean{x}-t_{\alpha/2}\frac{s}{\sqrt{n}} < \mu < \mean{x}+t_{\alpha/2}\frac{s}{\sqrt{n}}$. Calculate $\alpha$: $95 = 100(1-\alpha)$, $\alpha = 0.05$.
    $170\pm t_{0.025,24}\frac{5}{\sqrt{25}} = 167.9 < \mu < 172.1$. CI does not include 175, it is likely the average height has changed in the last 10 years.
    \rule{0.5\textwidth}{0.4pt}
    16 batteries for both type 1 and type 2. $\mean{x}_1=37.3$, $\mean{x}_2=40.5$, $s_1=1.9$, $s_2=2.1$. Assume batteries are normally distributed and assum $\sigma_1 = \sigma_2$.\newline
    (a) Find 95\% confidence interval for $\mu_1$. $\mu$ known, $\sigma$ unknown. 
    $37.3 - 2.131\frac{1.9}{\sqrt{16}} < \mu_1 < 37.3 + 2.131\frac{1.9}{\sqrt{16}} = 36.3 < \mu_1 < 38.3$.\newline
    (b) Find 90\% confidence interval for $\mu_2$. $\mu$ known, $\sigma$ unknown.
    $40.5 -1.753 \frac{2.1}{\sqrt{16}} < \mu_2 < 40.5 + 1.753 \frac{2.1}{\sqrt{16}} = 39.6 < \mu_2 < 41.4$\newline
    (c) From (a) and (b), which CI has a higher probability of containing the true mean?
    Since we don't know what the true mean is, it is impossible to tell which CI it is in.\newline
    (d) Find a 95\% confidence interval for $\mu_1-\mu_2$. $s_p^2=\frac{(n_1-1)s_1^2 + (n_2-1)s^2_2}{n_1+n_2-2} = 4.01$
    $(\mean{x}_1 - \mean{x}_2)-t_{0.025,30}s_p\sqrt{\frac{1}{n_1}+\frac{1}{n_2}} < \mu_1 - \mu_2 < (\mean{x}_1 - \mean{x}_2)+t_{0.025,30}s_p\sqrt{\frac{1}{n_1}+\frac{1}{n_2}}
    = -4.6 < \mu_1 - \mu_2 < -1.8$.\newline
    (e) From (d), what conclusions can you draw for the two types of batteries?.
    The CI only contains negative values, which suggests that battery type 2 might be better.
    \rule{0.5\textwidth}{0.4pt}
    \uline{Hypergeometric Distribution:} Choosing successful items. $h(x;N,n,k)=\frac{\binom{k}{x}\binom{N-k}{n-x}}
    {\binom{N}{n}}$. $max(0,n-(N-k))\leq x \leq min(n,k)$. x: num of successes. 
    N: num of items. n: num of selection. k: num of total successes. $\mu=\frac{nk}{N}$,
    $\sigma^2=\frac{N-n}{N-1}n\frac{k}{N}(1-\frac{k}{N})$\newline
    \uline{Negative Binomial Distribution:} Prob. the kth success will happen by the xth trial.
    $b^*(x;k,p)=\binom{x-1}{k-1}p^kq^{x-k}$. x: trial number. k: success number. p: prob. success.
    q: prob. failure.\newline
    \uline{Normal Distribution:} Bell curve. $n(x;\mu,\sigma)=\frac{1}{\sigma\sqrt{2\pi}}e^{-\frac{1}{2\sigma^2}(x-\mu)^2}$.
    x: select time. $\mu$: mean. $\sigma$: standard deviation.\newline
    \uline{CI on $\mu$, $\sigma^2$ known:} $\bar{x}-z_{\alpha/2}\frac{\sigma}{\sqrt{n}} < \mu < \bar{x}+z_{\alpha/2}\frac{\sigma}{\sqrt{n}}$.\newline
    \uline{CI for $\mu_1-\mu_2$, $\sigma_1^2$ and $\sigma_2^2$ known:} $(\bar{x}_1-\bar{x}_2)-z_{\alpha/2}\sqrt{\frac{\sigma_1^2}{n_1}+\frac{\sigma_2^2}{n_2}} < \mu_1-\mu_2 < (\bar{x}_1-\bar{x}_2)+z_{\alpha/2}\sqrt{\frac{\sigma_1^2}{n_1}+\frac{\sigma_2^2}{n_2}}$.\newline
    \uline{CI for $\mu_1-\mu_2$, $\sigma_1^2\not=\sigma_2^2$ and both unknown:} \\$(\bar{x}_1-\bar{x}_2)-t_{\alpha/2}\sqrt{\frac{s_1^2}{n_1}+\frac{s_2^2}{n_2}} < \mu_1-\mu_2 < (\bar{x}_1-\bar{x}_2)+t_{\alpha/2}\sqrt{\frac{s_1^2}{n_1}+\frac{s_2^2}{n_2}}$\newline
    
\end{multicols*}

\end{document}