\documentclass{article}
\usepackage{amsthm, amsmath}
\usepackage[margin=1in]{geometry}
\begin{document}
    \noindent\textbf{CS 375 Homework 1}\hfill Anchu A. Lee\\
    \noindent\today
    \\\\“I have done this assignment completely on my own. I have not copied it, nor have I given my solution to anyone else. I understand that if I am involved in plagiarism or cheating I will have to sign an official form that I have cheated and that this form will be stored in my official university record. I also understand that I will receive a grade of 0 for the involved assignment for my first offense and that I will receive a grade of “F” for the course for any additional offense.” 
    \\\\
    \begin{enumerate}
        \item Use the Master theorem to solve the following recurrences.
            \begin{enumerate}
                \item $T(n)=3T(n/4)+n$\\
                    $a = 3$, $b = 4$, $f(n) = n$\\
                    Case 3: $f(n) = \Theta(n^c)$ if $c=1$. \\
                    $log_{4}3 = 0.79248 < c$\\
                    $T(n) = \Theta(f(n)) = \Theta(n)$
                \item $T(n)=2T(n/4)+\sqrt{n}$log$(n)$\\
                    $a = 2$, $b = 4$, $f(n) = \sqrt{n}$log$(n)$\\
                    Case 2: $f(n)=\Theta(n^c$log$^kn)$ if $c= \frac{1}{2}$ and $k=0$\\
                    log$_{4}2 = 0.5$ so $c =$log$_ba$\\
                    $T(n)=\Theta(n^{0.5}$log$^1n = \Theta(\sqrt{n}$log$(n))$
                \item $T(n)=5T(n/2)+n^2$\\
                    $a = 5$, $b = 2$, $f(n) = n^2$\\
                    Case 1: $f(n) = \Theta(n^c)$ if $c = 2$\\
                    log$_{2}5=2.3219... > c$\\
                    $T(n) = \Theta(n^{log_{2}5}) = \Theta(n^{2.3218...})$
            \end{enumerate}
        \item Solve the recurrence 
            \[
                T(n) = 
                \begin{cases}
                    \Theta(1) & \text{for $n\leq 1$}\\
                    T(n/4)+T(3n/4)+n & \text{otherwise}
                \end{cases}
            \]
            using the recursion tree method. Draw the recursion tree and show the aggregate instruction counts for the following levels (0th, 1st, and last levels), and derive the Θ growth class for T(n) with justifications.
            \\\\\\\\\\\\\\\\\\\\\\\\\\\\\\\\\\\\\\\\\
        \item Use the substitution method to prove that $T(n)=T(n-1)+n\in O(n^2)$
        \item Assume that you are given an array of $n(n\geq 1)$ elements sorted in non-descending order. Design a \textit{ternary} search function that searches the array for a given element $x$ by applying the divide and conquer strategy.
        \item Develop a divide-and-conquer approach to selection (and hence a solution for the finding median problem). Hint: for any number $v$, imagine splitting list $S$ into three categories: elements smaller than $v$, those equal to $v$ (there might be duplicates), and those greater than $v$. 
    \end{enumerate}
\end{document}