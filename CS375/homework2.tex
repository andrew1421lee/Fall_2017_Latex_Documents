\documentclass{article}
\usepackage{amsthm, amsmath, listings}
\usepackage[margin=1in]{geometry}
\begin{document}
    \noindent\textbf{CS 375 Homework 1}\hfill Anchu A. Lee\\
    \noindent\today
    \\\\“I have done this assignment completely on my own. I have not copied it, nor have I given my solution to anyone else. I understand that if I am involved in plagiarism or cheating I will have to sign an official form that I have cheated and that this form will be stored in my official university record. I also understand that I will receive a grade of 0 for the involved assignment for my first offense and that I will receive a grade of “F” for the course for any additional offense.” 
    \\\\
    \begin{enumerate}
        \item Use the Master theorem to solve the following recurrences.
            \begin{enumerate}
                \item $T(n)=3T(n/4)+n$\\
                    $a = 3$, $b = 4$, $f(n) = n$\\
                    Case 3: $f(n) = \Theta(n^c)$ if $c=1$. \\
                    $log_{4}3 = 0.79248 < c$\\
                    $T(n) = \Theta(f(n)) = \Theta(n)$
                \item $T(n)=2T(n/4)+\sqrt{n}$log$(n)$\\
                    $a = 2$, $b = 4$, $f(n) = \sqrt{n}$log$(n)$\\
                    Case 2: $f(n)=\Theta(n^c$log$^kn)$ if $c= \frac{1}{2}$ and $k=0$\\
                    log$_{4}2 = 0.5$ so $c =$log$_ba$\\
                    $T(n)=\Theta(n^{0.5}$log$^1n = \Theta(\sqrt{n}$log$(n))$
                \item $T(n)=5T(n/2)+n^2$\\
                    $a = 5$, $b = 2$, $f(n) = n^2$\\
                    Case 1: $f(n) = \Theta(n^c)$ if $c = 2$\\
                    log$_{2}5=2.3219... > c$\\
                    $T(n) = \Theta(n^{log_{2}5}) = \Theta(n^{2.3218...})$
            \end{enumerate}
        \item Solve the recurrence 
            \[
                T(n) = 
                \begin{cases}
                    \Theta(1) & \text{for $n\leq 1$}\\
                    T(n/4)+T(3n/4)+n & \text{otherwise}
                \end{cases}
            \]
            using the recursion tree method. Draw the recursion tree and show the aggregate instruction counts for the following levels (0th, 1st, and last levels), and derive the $\Theta$ growth class for $T(n)$ with justifications.
            \\\\\\\\\\\\\\\\\\\\\\\\\\\\\\\\\\\\\\\\\
        \item Use the substitution method to prove that $T(n)=T(n-1)+n\in O(n^2)$
            \begin{proof}
                Assume that $T(n)=O(n^2)$. So then $T(n)\leq c\cdot n^2$ for some constant $c$.\\
                Assume $T(k)\leq ck^2$ for $k<n$. Prove $T(n)\leq cn^2$ by induction.
                \begin{align*}
                    T(n) = T(n-1)+n &\leq c\cdot(n-1)^2+n\\
                         &\leq c\cdot(n-1)(n-1)+n\\
                         &\leq c\cdot (n^2-2n+1) + n\\
                         &\leq cn^2-cn+c\leq cn^2
                \end{align*}
                Which holds provided $cn+c\geq 0$. Which is $cn\geq -c$. So $T(n)$ is in $O(n^2)$ as long as $c\geq 0$ and $n\geq 0$.
            \end{proof}
        \item Assume that you are given an array of $n$ $(n\geq 1)$ elements sorted in non-descending order. Design a \textit{ternary} search function that searches the array for a given element $x$ by applying the divide and conquer strategy.
            \begin{itemize}
                \item \textbf{Divide:} Grab an array index at $1/3$ of the array length ($a_1$) and at $2/3$ of the array length ($a_2$). That way the indexes split the array into thirds.
                \item \textbf{Conquer:} If the element $x$ is less than $A[a_1]$ then it must be in the subarray $A[0$ to $a_1]$. Otherwise if $x$ is greater than $A[a_1]$ and less than $A[a_2]$ then it must be in the subarray $A[a_1$ to $a_2]$ Lastly if $x$ is greater than $A[a_2]$ then it must be in the subarray $A[a_2$ to $n]$. Then recusievly search the subarray until $x$ is the value of $A[a_1]$ or $A[a_2]$.
                \item \textbf{Combine:} The final answer is the index found when the recursive function returns.
            \end{itemize}
            \begin{lstlisting}
    function ternarySearch(x, A, left, right) 
        a_1 = 1/3 * (right-left)  // first index
        a_2 = 2/3 * (right-left)  // second index
        if A[a_1] == x return a_1   // found x
        if A[a_2] == x return a_2

        // check left subarray
        if A[a_1] > x return ternarySearch(x, A, left, a_1-1)

        // check right subarray
        else if A[a_2] < x return ternarySearch(x, A, a_2+1, right)
        
        // check middle subarray
        else return ternarySearch(x, A, a_1+1, a_2-1)
            \end{lstlisting}
            The recursive time complexity of ternarySearch would be $T(n)=T(n/3)+\Theta(1)$. $n/3$ because the size of the array that needs to be searched is divided by three. Other functions of ternarySearch is trivial so happens over $\Theta(1)$\\
            Solve $T(n)=T(n/3)+\Theta(1)$ using the master theorem.\\
            $a = 1$, $b = 3$, $f(n) = \Theta(1)$\\
            Guess case 2: $f(n)=\Theta(n^c$log$^kn)$ is true for $c=0$ and $k=0$\\
            log$_31 =0= c$ so case 2 condition satisfied.\\
            Thus $T(n)=\Theta(n^0$log$^{k+1}n) = \Theta($log$n)$
        \item Develop a divide-and-conquer approach to selection (and hence a solution for the finding median problem). Hint: for any number $v$, imagine splitting list $S$ into three categories: elements smaller than $v$, those equal to $v$ (there might be duplicates), and those greater than $v$. 
    \end{enumerate}
\end{document}