\documentclass{article}
\usepackage{amsthm, amsmath}
\usepackage[margin=1in]{geometry}
    \begin{document}
        \noindent\textbf{CS 375 Homework 1}\hfill Anchu A. Lee
        \\\\“I have done this assignment completely on my own. I have not copied it, nor have I given my solution to anyone else. I understand that if I am involved in plagiarism or cheating I will have to sign an official form that I have cheated and that this form will be stored in my official university record. I also understand that I will receive a grade of 0 for the involved assignment for my first offense and that I will receive a grade of “F” for the course for any additional offense.” 
        \\\\
        \begin{enumerate}
            \item Given the pseudo code below for bubble sort:
                \begin{enumerate}
                    \item Let \verb|length[A] = n|. What is the count for \verb|BubbleSort(A)|? Show the steps necessary to derive your final answer. 
                        \begin{center}
                            \begin{tabular}{|c c c|}
                                \hline
                                Line & Cost & Times\\[0.5ex]
                                \hline\hline
                                1 & $c_1$ & n\\
                                \hline
                                2 & $c_2$ & $\sum_{i+1}^{n}t_i$\\
                                \hline
                                3 & $c_3$ & $\sum_{i+1}^{n}(t_i-1)$\\
                                \hline
                                4 & $c_4$ & 0 (best), $\sum_{i+1}^{n}(t_i-1)$ (worst) \\
                                \hline
                            \end{tabular}
                        \end{center}
                        $\sum_{line}^4($times $line$ executed$) = \text{instruction count}$\\
                    \item Show the worst case and best case time complexity in term of instruction counts.\\
                    Worst case scenario:\\
                    $n + \sum_{i+1}^{n}t_i + \sum_{i+1}^{n}(t_i-1) + \sum_{i+1}^{n}(t_i-1)$\\
                    Best case scenario:\\
                    $n + \sum_{i+1}^{n}t_i + \sum_{i+1}^{n}(t_i-1)$
                \end{enumerate}
            \item Fill in all the missing values.
                \begin{center}
                    \begin{tabular}{|c | c | c | c | c|}
                        \hline
                        $f(n)$ & $g(n)$ & $f(n)=O(g(n))$ & $f(n)=\Omega(g(n))$ & $f(n)=\Theta(g(n))$\\
                        \hline
                        \hline
                        $n^{2.125}$ & $n^2\text{lg}n$ & No & Yes & No\\
                        \hline
                        $\sqrt{n}$ & $n$ & Yes & No & No\\
                        \hline
                        $n!$ & $(n+1)!$ & Yes & No & No\\
                        \hline
                        $2^{n/2}$ & $2^n$ & Yes & No & No\\
                        \hline
                        $\sum_{i=1}^{n}i= \frac{n^2+n}{2}$ & $n^2$ & Yes & Yes & Yes\\
                        \hline
                        $\sum_{i=0}^{n-1}4^i=\frac{1}{3}(4^n-1)$ & $n4^{n-1}$ & No & Yes & No\\
                        \hline
                    \end{tabular}
                \end{center}
            \item Order the functions below by increasing growth rate.
                \begin{center}
                    $n^n$, $n$ln$n$, $n^\epsilon (0<\epsilon<1)$, $2^{\text{ln}n}$, $\text{ln}n$, $10$, $n!$, $2^n$\\
                    $10$, ln$n$, $2^{\text{ln}n}$, $n^\epsilon (0<\epsilon<1)$, $n$ln$n$, $2^n$, $n^n$, $n!$
                \end{center}
            \item Let $f(n)$ and $g(n)$ be asymptotically positive functions. Prove or show a counter example for each of the following conjectures.
                \begin{enumerate}
                    \item $f(n)\in O(g(n))$ implies $2^{f(n)}\in O(2^{g(n)})$\\
                        False.
                        \begin{proof}
                            Consider $f(n) = 2n$ and $g(n) = n$.
                            $2n \in O(n)$\\
                            Let's then assume $2^{2n}\in O(2^{n})$. By definition of big-O, there must exist some constant $c$, $n_0$
                             such that $0\leq 2^{2n}\leq c\cdot 2^n$, $\forall n\geq n_0$. Let's try to prove $2^{2n}\leq c\cdot 2^n$ for some constant $c$.
                            \begin{align*}
                                2^{2n}&\leq c\cdot 2^n\\
                                (2^n)^2&\leq c\cdot 2^n\\
                                2^n&\not\leq \sqrt{c}\cdot 2^{n/2}
                            \end{align*}                      
                            As $c$ is constant, as $n$ approaches $\infty$ the LHS will become greater than the RHS. Therefore the implication is not true.
                        \end{proof}
                    \item $f(n)\in O(g(n))$ implies $g(n)\in\Omega(f(n))$\\
                        True.
                        \begin{proof}
                            The definition of $f(n)\in O(g(n))$ says there must exist a constant $c$ such that $0\leq f(n)\leq c\cdot g(n)$.\\
                            The definition of $g(n)\in\Omega(f(n))$ says there must exist a constant $c$ such that $0\leq c\cdot f(n)\leq g(n)$.\\
                            Therefore $f(n)\in O(g(n)) \Leftrightarrow g(n)\in\Omega(f(n))$

                        \end{proof}
                \end{enumerate}
            \item Prove $n^2-3n-20\in\Theta(n^2)$ using the original definition of $\Theta$
                \begin{proof}
                    By definition of $\Theta$, \\$\Theta(n^2) = \{n^2-3n-20 \text{ : there exist positive constants } c_1 \text{, } c_2 \text{, } n_0 \text{ such that: }\\ 
                        \text{\hspace{1.5cm}}0\leq c_1(n^2)\leq n^2-3n-20\leq c_2(n^2), \forall n\geq n_0\}$
                    \\\\
                    Let $c_1 = \frac{1}{2}$, $c_2 = 1$ for all $n \geq 10$.
                    First, need to show that $\frac{1}{2}n^2\leq n^2-3n-20$\\
                    Base case: $n=10$
                    \begin{align*}
                        \frac{1}{2}(10)^2 &\leq (10)^2-3(10)-20\\
                        \frac{100}{2} &\leq 100-30-20\\
                        50 &= 50
                    \end{align*}
                    IH: Assume $\frac{1}{2}k^2 \leq k^2-3k-20$, $k\geq 10$\\
                    Prove true for $k+1$.
                    \begin{align*}
                        \frac{1}{2}(k+1)^2 &\leq (k+1)^2-3(k+1)-20\\
                        \frac{1}{2}(k+1)^2 &\leq k^2+2k+1-3k-3-20\\
                        \frac{1}{2}(k+1)^2 &\leq k^2-k-22 \\
                        \frac{1}{2}(k+1)^2 &\leq k^2-k-22-2k+2k+2-2 \\
                        \frac{1}{2}(k+1)^2 &\leq k^2-3k-20+2k-2 \\
                        \frac{1}{2}(k+1)^2 &\leq \frac{1}{2}k^2+2k-2 &&\text{Induction step}\\
                        \frac{1}{2}k^2 + k +\frac{1}{2} &\leq \frac{1}{2}k^2+2k-2 \\ 
                        \frac{1}{2} &< k-2
                    \end{align*}
                    Now, show that $n^2-3n-20\leq n^2$\\
                    Base case: $n=10$
                    \begin{align*}
                        (10)^2-3(10)-20 &\leq (10)^2 \\
                        100-30-20 &\leq 100\\
                        50 &\leq 100
                    \end{align*}
                    IH: Assume $k^2-3k-20\leq k^2$ for $k\geq 10$\\
                    Prove true for $k+1$.
                    \begin{align*}
                        (k+1)^2-3(k+1)-20 &\leq (k+1)^2 \\
                        k^2+2k+1-3k-3-20 &\leq (k+1)^2 \\
                        k^2-k-22 &\leq (k+1)^2 \\
                        k^2-k-22-2k+2k+2-2 &\leq (k+1)^2 \\
                        k^2-3k-20+2k-2 &\leq (k+1)^2 \\
                        k^2+2k-2 &\leq (k+1)^2 &&\text{Induction step}\\
                        k^2+2k-2 &\leq k^2+2k+1 \\
                        -2 &\leq 1
                    \end{align*}
                    This shows that $n^2-3n-20\in\Theta(n^2)$ is true with $c_1 = \frac{1}{2}$, $c_2 = 1$ for all $n \geq 10$
                \end{proof}
            \item Disprove $n^3\in O(n^2)$ using the original definition of $O$.
                \begin{proof}
                    By definition of $O$,\\$O(n^2) = \{n^3 \text{ : there exist positive constants } c \text{, } n_0 \text{ such that: }\\ 
                    \text{\hspace{1.5cm}}0\leq n^3\leq c(n^2), \forall n\geq n_0\}$\\
                    Need to show that $n^3\leq c(n^2)$ is never true. Let c be any constant number.
                    \begin{align*}
                        n^3 &\leq c(n^2)\\
                        1 &\leq \frac{c}{n} \\
                        n &\leq c
                    \end{align*}
                    As $n$ approaches infinity it cannot be bounded by a constant. Thus $n^3\not\in O(n^2)$.
                \end{proof}
            \item Prove $n=\omega($lg$n^2)$ using limit.
                \begin{proof}
                    If $n=\omega($lg$n^2)$, then the limit $\lim_{n\to\infty}\frac{n}{lgn^2}$ approaches $\infty$.
                    \begin{align*}
                        \lim_{n\to\infty}\frac{n}{\text{lg}n^2}&\\
                        \lim_{n\to\infty}\frac{1}{\frac{2}{n}}&\\
                        \lim_{n\to\infty}\frac{n}{2}& \Rightarrow \infty
                    \end{align*}
                \end{proof}
            \item Prove $n^a=\omega($lg$^kn)$, where $k>0$, $a>0$, using limit.
                \begin{proof}
                    If $n^a=\omega($lg$^kn)$ then the limit $\lim_{n\to\infty}\frac{n^a}{lg^kn}$ approaches $\infty$.
                    \begin{align*}
                        \lim_{n\to\infty}\frac{n^a}{\text{lg}^kn}&\\
                        \lim_{n\to\infty}\frac{an^{a}}{k\text{lg}^{k-1}(n)}&
                    \end{align*}
                \end{proof}
        \end{enumerate}
    \end{document}