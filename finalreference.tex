\documentclass[9pt]{article}
\usepackage{amsthm, amsmath, extsizes, multicol, amsfonts, enumitem, tabularx}
\usepackage[normalem]{ulem}
\usepackage[margin=0.25in]{geometry}
\newcommand*\mean[1]{\bar{#1}}
\newcommand*\median[1]{\tilde{#1}}

\setlength{\multicolsep}{2pt}
\begin{document}
\noindent\textbf{Midterm 1: Chapters 1 to 4}
\addtolength{\tabcolsep}{-2pt}
\begin{multicols*}{2}
    \noindent\begin{tabular*}{0.5\textwidth}{c c c c c c c c c c}
        \hline
        1 & 2 & 3 & 4 & 5 & 6 & 7 & 8 & 9 & 10\\
        22.43 & 10.25 & 23.71 & 21.77 & 22.11 & 18.71 & 19.77 & 20.33 & 20.17 & 21.12\\
        \hline
    \end{tabular*}
    (a) $\mean{x}=\frac{\Sigma_{i=1}^{10}x_i}{10} = 20.037$. 
    $\median{x}=\frac{20.33 + 21.12}{2} = 20.725$\newline
    (b) $s_{x}^{2}=\frac{\Sigma(x_i-\mean{x})^2}{n-1} = 13.935$. 
    Split into 4 sections, numbers seperating are quartiles. 
    Last minus first is IQR $=22.11-19.77 = 2.34$\newline
    (c) Trimmed mean of 10\%: remove 10\% from highest and lowest. $= 20.801$.
    Close to median but more than mean; data is slightly skewed to the left.\newline
    (d) Set decimal point to $\mid$.\newline
    (e) Away from Q1 and Q3 by $1.5\cdot$IQR are outliers. Left dot is minimum, start of
    box is Q1, middle line is median, end of box is Q3, last dot is maximum.
\end{multicols*}

\end{document}