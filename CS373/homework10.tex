\documentclass{article}
\usepackage{amsthm, amsmath, graphicx, enumitem, tikz-qtree}
\usepackage[margin=0.5in]{geometry}
\begin{document}
    \noindent\textbf{CS 373 Homework 10}\hfill Anchu A. Lee\\
    \noindent\today\\
    \begin{enumerate}
        \item Show that $EQ_{CFG}$ is undecidable.\newline

        was in class
        \item Show that $EQ_{CFG}$ is co-Turing-recognizable.\newline

        was in class
        \item Find a match in the following instance of the Post
        Correspondence Problem.
        \begin{center}
            $\{[\frac{ab}{abab}], [\frac{b}{a}], [\frac{aba}{b}],
            [\frac{aa}{a}]\}$
        \end{center}
        was in class
        \item Show that the Post Correspondence Problem is decidable 
        over the unary alphabet $\Sigma = \{1\}$.\newline

        was in class
        \item  In the \textit{silly Post Correspondence Problem, SPCP}, 
        the top string in each pair has the same length as the bottom 
        string. Show that the \textit{SPCP} is decidable.\newline

        was in class
        \item Show that $A$ is Turing-recognizable iff $A\leq_{m} A_{TM}$\newline

        was in class
        \item \textbf{Rice’s theorem}. Let P be any nontrivial property of the language of a Turing
        machine. Prove that the problem of determining whether a given Turing machine’s
        language has property $P$ is undecidable.\newline
        In more formal terms, let $P$ be a language consisting of Turing machine descriptions 
        where $P$ fulfills two conditions. First, $P$ is nontrivial - it contains some, but
        not all, TM descriptions. Second, $P$ is a property of the TM’s language—whenever
        $L(M_1) = L(M_2)$, we have $\langle M_1 \rangle \in P $ iff $ \langle M_2 \rangle \in P$. 
        Here, $M_1 and M_2$ are any TMs. Prove that $P$ is an undecidable language.\newline

        Assume for the sake of contradiction that $P$ is a decidable language satisfying the
        properties and let $R_P$ be a TM that decides $P$. We show how to decide $A_{TM}$ using
        $R_P$ by constructing TM $S$. First, let $T_{\emptyset}$ be a TM that always rejects, 
        so $L(T_{\emptyset})=\emptyset$.
        You may assume that $\langle T_{\emptyset} \rangle \not\in P$ without loss of generality 
        because you could proceed with $P$ instead of $P$ if $\langle T_{\emptyset} \rangle \in P$. 
        Because $P$ is not trivial, there exists a TM $T$ with $\langle T \rangle \in P$. Design $S$ 
        to decide $A_{TM}$ using $R_P$’s ability to distinguish between $T_{\emptyset}$ and $T$.\newline

        $S=$"On input $\langle M,w\rangle$:
        \begin{enumerate}[label = \arabic*., topsep=0pt]
            \item Use $M$ and $w$ to construct the following TM $M_w$.\newline
            $M_w=$ "On input x:
            \begin{enumerate}[label = \arabic*., topsep=0pt]
                \item Simulate $M$ on $w$. If it halts and rejects, reject. If it accepts, proceed to
                stage 2.
                \item Simulate $T$ on $x$. If it accepts, accept."
            \end{enumerate}
            \item Use TM $R_P$ to determine whether $\langle M_w \rangle\in P$. If YES, accept. if NO,
            reject."
        \end{enumerate}
        TM $M_w$ simulates $T$ if $M$ accept $w$. Hence $L(M_w)$ equals $L(T)$ if $M$ accepts $w$ and 
        $\emptyset$ otherwise. Therefore $\langle M_w \rangle \in P$ iff $M$ accepts $w$

        \item Let
        \begin{center}
            $f(x) = 
            \begin{cases} 
                3x+1 & \text{for odd x}\\
                x/2 & \text{for even x}
            \end{cases}$
        \end{center}
        for any natural number $x$. If you start with an integer $x$ and iterate $f$, you obtain a
        sequence, $x,f(x),f(f(x)),...$. Stop if you ever hit 1. For example, if $x = 17$, you
        get the sequence 17, 52, 26, 13, 40, 20, 10, 5, 16, 8, 4, 2, 1. Extensive computer
        tests have shown that every starting point between 1 and a large positive integer
        gives a sequence that ends in 1. But the question of whether all positive starting
        points end up at 1 is unsolved; it is called the $3x + 1$ problem.
        Suppose that $A_{TM}$ were decidable by a TM $H$. Use $H$ to describe a TM that is
        guaranteed to state the answer to the $3x + 1$ problem.\newline

        was in class
        \item Let $T = \{(i,j,k) \mid i,j,k \in N\}$. Show that $T$ is countable.\newline

        was in class
        \item Review the way that we define sets to be the same size in Definition 4.12 (page 203).
        Show that “is the same size” is an equivalence relation.\newline

        was in class
    \end{enumerate}
\end{document}