\documentclass{article}
\usepackage{amsthm, amsmath, graphicx, enumitem, tikz-qtree}
\usepackage[margin=0.5in]{geometry}
\begin{document}
    \noindent\textbf{CS 373 Homework 8}\hfill Anchu A. Lee\\
    \noindent\today\\
    \begin{enumerate}
        \item This exercise concerns TM $M_2$ , whose description and state diagram appear in 
        Example 3.7. In each of the parts, give the sequence of configurations that $M_2$ 
        enters when started on the indicated input string.
        \begin{enumerate}[label = (\alph*) ]
            \item 0.\newline
            \begin{tabular}{c c c c c}
                $q_10$\textvisiblespace & \textvisiblespace$q_2$\textvisiblespace 
                & \textvisiblespace\textvisiblespace$q_{accept}$
            \end{tabular}
            \item 000.\newline
            \begin{tabular}{c c c c c}
                $q_1000$\textvisiblespace & \textvisiblespace$q_200$\textvisiblespace 
                & \textvisiblespace\textvisiblespace$q_30$\textvisiblespace 
                & \textvisiblespace\textvisiblespace$0q_4$\textvisiblespace
                & \textvisiblespace\textvisiblespace 0\textvisiblespace$q_{reject}$
            \end{tabular}
                
        \end{enumerate}
        \item  This exercise concerns TM $M_1$ , whose description and state diagram appear in 
        Example 3.9. In each of the parts, give the sequence of configurations that $M_1$ 
        enters when started on the indicated input string.
        \begin{enumerate}[label = (\alph*) ]
            \item 1\#1.\newline
            \begin{tabular}{c c c c c c c c c}
                $q_1 1\#1$\textvisiblespace & x$q_3\#1$\textvisiblespace & x\#$q_51$\textvisiblespace
                & x$q_6$\#x\textvisiblespace & $q_7$x\#x\textvisiblespace & x$q_1$\#x\textvisiblespace 
                & x\#$q_8$x\textvisiblespace & x\#x$q_8$\textvisiblespace & x\#x\textvisiblespace$q_{accept}$
            \end{tabular}
            \item 1\#\#1.\newline
            \begin{tabular}{c c c c c c c c c}
                $q_1 1\#\#1$\textvisiblespace & x$q_3$\#\#1\textvisiblespace 
                & x\#$q_5$\#1\textvisiblespace & x\#\#$q_{reject}$1\textvisiblespace
            \end{tabular}
        \end{enumerate}
        \item Describe a Turing machine, sequence of steps, that recognizes 
        $\{ w \mid w $ is an element of $ \{a, b, 
        c\}* $ such that the number of $a$'s in $w <$ the number of $b$'s in $w$ and the number 
        of $a$'s in $w =$ the number of $c$'s in $w \}$
        \item Show the equivalent transitions for a 2-PDA for the 
        Turing machine transitions $(q_i, X) \rightarrow (q_j, A, L)$ and $(q_i, X) \rightarrow 
        (q_j, A, R)$ (in state $q_i$ read $X$, write $A$, and move left or right and transition to 
        state $q_j$). The transitions for a 2-PDA are of the form $(q_i, X, S_1, S_2) \rightarrow
        (q_j, T_1, T_2)$ (in state $q_i$, read $X$, pop $S_1$ from stack 1, pop $S_2$ from stack 2, 
        transition to state $q_j$, push $T_1$ onto stack 1 and push $T_2$ onto stack 2). You don't 
        have to prove the transitions are equivalent, just tell me what they are.
        \item Give implementation-level descriptions of Turing machines that decide the following 
        languages over the alphabet $\{0,1\}$. $\{w\mid w $ does not contain twice as many 0's as 
        1's$\}$
        \item Prove the class of Turing recognizable languages is closed under the union operation 
        (construction and proof)
        \item Prove the class of decidable languages is closed under concatenation (construction 
        and proof)
        \item Prove the class of decidable languages is closed under intersection (construction 
        and proof)
        \item Prove the class of Turing recognizable languages is closed under the star operation 
        (construction and proof)
        \item Show that a language is decidable iff some enumerator enumerates the language in
        the standard string order.
    \end{enumerate}
\end{document}
