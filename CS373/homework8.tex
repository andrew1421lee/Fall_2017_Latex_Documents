\documentclass{article}
\usepackage{amsthm, amsmath, graphicx, enumitem, tikz-qtree}
\usepackage[margin=0.5in]{geometry}
\begin{document}
    \noindent\textbf{CS 373 Homework 8}\hfill Anchu A. Lee\\
    \noindent\today\\
    \begin{enumerate}
        \item This exercise concerns TM $M_2$ , whose description and state diagram appear in 
        Example 3.7. In each of the parts, give the sequence of configurations that $M_2$ 
        enters when started on the indicated input string.
        \begin{enumerate}[label = (\alph*) ]
            \item 0.\newline
            \begin{tabular}{c c c c c}
                $q_10$\textvisiblespace & \textvisiblespace$q_2$\textvisiblespace 
                & \textvisiblespace\textvisiblespace$q_{accept}$
            \end{tabular}
            \item 000.\newline
            \begin{tabular}{c c c c c}
                $q_1000$\textvisiblespace & \textvisiblespace$q_200$\textvisiblespace 
                & \textvisiblespace\textvisiblespace$q_30$\textvisiblespace 
                & \textvisiblespace\textvisiblespace$0q_4$\textvisiblespace
                & \textvisiblespace\textvisiblespace 0\textvisiblespace$q_{reject}$
            \end{tabular}
                
        \end{enumerate}
        \item  This exercise concerns TM $M_1$ , whose description and state diagram appear in 
        Example 3.9. In each of the parts, give the sequence of configurations that $M_1$ 
        enters when started on the indicated input string.
        \begin{enumerate}[label = (\alph*) ]
            \item 1\#1.\newline
            \begin{tabular}{c c c c c c c c c}
                $q_1 1\#1$\textvisiblespace & x$q_3\#1$\textvisiblespace & x\#$q_51$\textvisiblespace
                & x$q_6$\#x\textvisiblespace & $q_7$x\#x\textvisiblespace & x$q_1$\#x\textvisiblespace 
                & x\#$q_8$x\textvisiblespace & x\#x$q_8$\textvisiblespace & x\#x\textvisiblespace$q_{accept}$
            \end{tabular}
            \item 1\#\#1.\newline
            \begin{tabular}{c c c c c c c c c}
                $q_1 1\#\#1$\textvisiblespace & x$q_3$\#\#1\textvisiblespace 
                & x\#$q_5$\#1\textvisiblespace & x\#\#$q_{reject}$1\textvisiblespace
            \end{tabular}
        \end{enumerate}
        \item Describe a Turing machine, sequence of steps, that recognizes 
        $\{ w \mid w $ is an element of $ \{a, b, c\}^* $ such that the number of $a$'s in $w <$ 
        the number of $b$'s in $w$ and the number of $a$'s in $w =$ the number of $c$'s in $w \}$
        \begin{enumerate}[label = (\arabic*)]
            \item Place symbol at the left side of tape
            \item Scan right for $a$, if found: mark it, else: go to step 6
            \item Rewind
            \item Scan right for $b$, if found: mark it, else: Halt and Reject ($a$ must be $< b$)
            \item Rewind and go to step 2.
            \item Rewind
            \item Scan right for $a'$, if found: mark it, else: go to step 11
            \item Rewind
            \item Scan right for $c$, if found: mark it, else: Halt and Reject ($a$ must be $= c$)
            \item go to step 6.
            \item Scan right for $c$, if found: Halt and Reject, else: Halt and Accept.
        \end{enumerate}
        \item Show the equivalent transitions for a 2-PDA for the 
        Turing machine transitions $(q_i, X) \rightarrow (q_j, A, L)$ and $(q_i, X) \rightarrow 
        (q_j, A, R)$ (in state $q_i$ read $X$, write $A$, and move left or right and transition to 
        state $q_j$). The transitions for a 2-PDA are of the form $(q_i, X, S_1, S_2) \rightarrow
        (q_j, T_1, T_2)$ (in state $q_i$, read $X$, pop $S_1$ from stack 1, pop $S_2$ from stack 2, 
        transition to state $q_j$, push $T_1$ onto stack 1 and push $T_2$ onto stack 2). You don't 
        have to prove the transitions are equivalent, just tell me what they are.\newline\newline
        $(q_i, X) \rightarrow (q_j, A, L)$ = $(q_i, X, \epsilon, X) \rightarrow (q_j, A, \epsilon)$
        read/pop from the right stack and push to the left stack.\newline
        $(q_i, X) \rightarrow (q_j, A, R)$ = $(q_i, X, X, \epsilon) \rightarrow (q_j, \epsilon, A)$
        read/pop from the left stack and push to the right stack.
        \item Give implementation-level descriptions of Turing machines that decide the following 
        languages over the alphabet $\{0,1\}$. $\{w\mid w $ does not contain twice as many 0's as 
        1's$\}$
        \begin{enumerate}[label=(\arabic*)]
            \item Place symbol at left side of tape
            \item Rewind            
            \item Scan right for 1, if found: mark it, else: go to step 9
            \item Rewind
            \item Scan right for 0, if found: mark it, else: Halt and Accept
            \item Rewind
            \item Scan right for 0, if found: mark it, else: Halt and Accept
            \item Go to step 2.
            \item Rewind
            \item Scan right for $0'$, if found: Halt and Reject, else: Halt and Accept
        \end{enumerate}
        \item Prove the class of Turing recognizable languages is closed under the union operation 
        (construction and proof)
        \begin{proof}
            Let $M_1$ and $M_2$ be two Turing machines that recognize languages $L_1$ and $L_2$
            respectfully. Then let $M_3$ be a machine that will run input $w$ alternately between 
            machines $M_1$ and $M_2$. If a machine accepts, $M_3$ accepts. If both 
            machines reject then $M_3$ rejects. As $w\in L_1\cup L_2$, the string can be $w\in L_1$,
            then the $M_1$ portions of $M_3$ will accept it. If the string is $w\in L_2$, then the
            $M_2$ portions of $M_3$ will accept it. If the string is $w\not\in L_1\cup L_2$ then 
            $w\not\in L_1$ and $w\not\in L_2$ and so $M_3$ will not accept $w$. Therefore $M_3$ 
            recognizes $L_1\cup L_2$.
        \end{proof}
        \item Prove the class of decidable languages is closed under concatenation (construction 
        and proof)
        \begin{proof}
            Let $M_1$ and $M_2$ be two Turing machines that recognize languages $L_1$ and $L_2$
            respectfully. Then let $M_3$ be a machine that will run input $w$ on $M_1$ and $M_2$
            by splitting $w$ into every possible two parts. If both machines accept then $M_3$ 
            accepts. If not, then the $M_3$ continues to the next two substrings.That means every 
            possible combination of two substrings of the string $w$ will be run through $M_1$ 
            and $M_2$. When all substrings are tried and did not reach an accepting state, then 
            reject $w$. That way the $w$ must be $L_1\circ L_2$ as the first substring is in 
            $M_1$ and the second substring will be accepted by $M_2$. Otherwise $w$ will be rejected.
        \end{proof}
        \item Prove the class of decidable languages is closed under intersection (construction 
        and proof)
        \begin{proof}
            Let $M_1$ and $M_2$ be two Turing machines that recognize languages $L_1$ and $L_2$
            respectfully. Then let $M_3$ be a machine that will run input $w$ on $M_1$ then $M_2$.
            If both machines reject, then $w$ is not in the languages of $L_1\cap L_2$. If one or
            more machines accepts then $w$ is in the language.
        \end{proof}
        \item Prove the class of Turing recognizable languages is closed under the star operation 
        (construction and proof)
        \begin{proof}
            Let $M_1$ be a Turing machine that recognizes the language $L_1$ and $w$ be a string 
            of the form $L_1^*$. Then have $M_2$ be a machine that splits the input into individual 
            cuts of the input. e.g. $s = s_1s_2s_3...s_n$ for $n$ can be from 0 to the length of 
            $s$. $M_2$ then runs each substring into $M_1$. If it rejects then $M_2$ tries the next
            cuts of the string. If $M_2$ accepts for all cuts of a string, then the language is 
            accepted. As $M_2$ tries every possible split for the input, it will eventually find
            the right match for $L_1$ and therefore recognizes $L_1^*$.
        \end{proof}
        \item Show that a language is decidable iff some enumerator enumerates the language in
        the standard string order.
        \begin{proof}
            $(\Rightarrow)$ There must exist a TM $M$ that decides $L$, a decidable language. $M$
            can reconstruct the enumerator $E$. Define a TM $D$ that: (ignores input)
            \begin{enumerate}
                \item Set an iterator $i$ to one of $1, 2, 3,...$
                \item Get the $i$th string of the languages $L$, $w_i$
                \item Run machine $M$ with string $w_i$. Write $w_i$ to tape if $M$ accepts.
            \end{enumerate}
            $D$ outputs all strings in $L$ in standard string order.\newline
            $(\Leftarrow)$ For a language $L$ with enumerator $E$, constructor a TM $M$ for $L$
            which decides the language. $M$ then can be run with input $w$:
            \begin{enumerate}
                \item Run through $E$ and save the strings outputted to tape as $w_1,w_2,w_3...$
                \item Check the tape against the input string $w$. If $w = w_i$ for some $i$, then
                halt and accept.
                \item If the machine encounters a string $w_i$ from the $E$ that comes after the input
                string $w$ in standard string order, then halt and reject.
            \end{enumerate}
        \end{proof}
    \end{enumerate}
\end{document}
